\section{厨房労働者の雇用形態と人件費負担について}\label{sec:tyuboin}

	現在, 熊野寮食堂の厨房に勤務する5名の栄養士・調理員のうち, 調理員2名については大学雇用ではなく, 寮生が人件費と労働保険料を負担している. 我々はこの現状の問題性を見失ってはならない.

		\subsection{「受益者負担の原則」とそれを否定する熊野寮の立場}

	「受益者負担の原則」とは「利益を得ている人」が金を払えという考え方である. これは60年代から国が押し出してきたものである. 教育について言えば, 「教育を受ければ将来良い職について金持ちになるから, 将来の自分への投資として金を払え」というような論法となる.

	それに対して, 社会の維持発展のために教育は必要であるという事実から, 社会に必要なものは社会全体で保障すべきであるという考え方がある. 社会全体が利益を得ているから社会全体で金を賄うというものである.
	つまり「受益者負担の原則」を否定する考え方である. 熊野寮自治会はこの立場で運動し, 国・当局との攻防を繰り広げて今にいたる.

	こう考えれば, 調理員の人件費についても本来公費で賄われるべきであり, つまり京都大学が調理員の雇用を保障すべきである.

	歴史を見れば, 寮自治会が運動した結果, 1969年7月以降, 調理員全員が京都大学に公務員として雇用されていた. 大学当局すらも, 国策である「受益者負担の原則」に反対する立場に立たせたのである.


		\subsection{食堂運営会とは}

		食堂運営会(正式名称:京都大学熊野寮食堂運営会) とは, 熊野寮食堂の厨房に勤務する5名の栄養士・調理員のうち, 寮生が人件費と労働保険料を負担している調理員2名を雇用し, その労働条件を維持・改善するためにつくられた組織である. 会長には京都大学副学長が着任し, 全寮生が会員ということになっている.

		60年代の時点では, 熊野寮食堂の厨房調理員は全員公務員(つまり大学が雇用主) だったが, 1979年12月, 大学側が国が推し進める受益者負担の原則や新々寮四条件\footnote{文科省の認める学寮の条件(新々寮4条件),
\begin{enumerate}
\item 全部屋個室
\item 食堂なし(集会スペースをつくらない)
\item 負担区分全面適用(居住に関しては個人と大学の間の契約関係に依拠)
\item 入退寮選考権は大学持ち(同上)
\end{enumerate}}
公務員削減政策を理由に, 調理員をそれまでと同じ条件では(つまり公務員としては)雇わないことを一方的に宣言した. それ以降4度の臨時職員の後任不補充が行われ, 抗議の甲斐もなく, 結果として寮生側は自分たちで人件費を払ってパート(時間雇用) 調理員2名を雇うことになったのである.

		さらに, 大学との間で誰がそのパート調理員の労働関係上の雇用主となるべきかをめぐって見解が合わず\footnote{寮生側は学生と職員の福利厚生に責任のある大学に全面的な雇用責任があるとし, 大学側は実質的に人件費を支払っている寮生にあるとした. }, 交渉が長期化した. その結果, 雇用主が不明確という理由で寮生負担パート調理員は長いこと労働保険にも正式に加入することができなかった. 労働保険(雇用保険・労災保険) への加入は全ての労働者に与えられた権利であるが, 寮生負担パート調理員はその労働者にとっての基本的な権利すら許されないまま放置されていたのである.

		そこでこの人権侵害的な状況を改善するため, 寮生側から大学に対し, 「食堂運営会」を設立し, その組織を寮生負担パート調理員の雇用主とすることを提案したのである. 取り決めとしては, 「人件費, 保険料は今までどおり寮生が払うこと」「大学側は労災時の補償責任を負うこと. また, 大学副学長は会長に就任すること」という形で, 大学と寮生が運営会雇用調理員の雇用責任を折半して負う形であった. この提案は認められ, 2005年4月から食堂運営会は発足し, 食堂運営会雇用調理員(旧寮生負担パート) は労働保険に加入することができた. 確約によると, 寮生は食堂運営会雇用調理員(旧寮生負担パート) 2名の人件費・保険料を負担し, 大学側は労働災害発生時の補償責任を負うことになっている.

		食堂運営会の今後の課題は, 食堂運営会雇用調理員さんの現在の労働条件を維持し, 働きやすい労働環境を守っていくことである. そのため, 日頃の調理員さんとのコミュニケーションはもちろん, 年2回開かれる食堂運営会総会や毎年の会長指名(会長の任期は1年) , 副学長の代替わり毎に行なう団交・確約を確実にこなさなければならない. これよりもう一段階進んだ課題としては, 調理員さんに対する人件費・保険料支払いも含む完全な雇用責任を大学に認めさせ, 大学を雇用主にすること(つまり労働条件の「正常化」) がある.

		低価で生活できる学寮の意義を放棄し, 寮生の共同空間を破壊することで, 寮生間の自由なコミュニケーションを否定し, 個々人を分断管理(集団として行動できないように) するものであると同時に, 大学, 社会に対して批判的な者を恣意的に退寮させるなど, 経済的弱者を切り捨てた上での徹底管理を狙ったものであるとして全国の自治寮は長年これに抗議してきた.

		\begin{shadebox}
      \textbf{学寮不要の文部省方針} \\
      1971年の中央教育審議会答申で, 文部省は学寮を「紛争の根源地」と断定, その教育的意義を否定した. これに基づいて多くの学寮で, 水光熱費の徴収や入退寮権を大学当局が把握していった. 大阪大学, 岡山大学などでは, 大学当局が反対する寮生を機動隊の力を借りて抑圧し, 自治寮の廃寮化を進めていった.
      
      \tatespace
      自治寮は, かつては全国にあったが, 次々と廃寮化, 管理化されている.
      \begin{itemize}
        \item 東大駒場寮2001年廃寮化
        \item 東北大有朋寮2003年管理寮化
        \item 富山大新樹寮2011年管理寮化
        \item 東京芸大石神井寮2014年廃寮化
      \end{itemize}
      など
  \end{shadebox}
