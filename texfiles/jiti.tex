\section{自治とは何か}\label{sec:jiti}

 	みなさんこんにちは. みなさんがこれから, 入られるであろうと期待する熊野寮は「自治寮」という, 寮生で「自治」を行っている寮です. 現在でもこのような「自治」を行っている学生寮は, 全国的に見てもほとんどありません. ここでは, 熊野寮の「自治」について, 紹介したいと思います.
 	
  「自治」とは何なのでしょうか. 辞書には「自分たちのことを自ら処理すること. 特に, 地方公共団体や大学がその範囲での行政・事務を自主的に行うこと(岩波国語辞典) 」と書いてあります. すこし幅が広すぎてよくわかりませんね.

  ですが, 熊野寮の自治とは非常に単純です. それは「熊野寮のことについては熊野寮生が取り決め運営する」という, ただそれだけのことです.「そんなことわざわざ確認しなくてもあたりまえ」という人がいるかもしれません. しかしながら, 熊野寮が誇る「自治」のすばらしさは, それをとことん追求していることにあります.

  たとえば, 現在政府は学生に学生寮の入退寮権(どのような学生を寮に入れるか判断する権利) を認めていません. 従って, 全国のほとんどの大学では学生寮には大学が許可した学生のみが入寮するようになっています. では, どうして熊野寮では入退寮権を学生組織である熊野寮自治会が持っているのでしょうか. それは, 熊野寮生が「自分たちのことは自分たちで決める」と国や大学の決定に対し闘って, 勝ち取ったからです.

  また, もともと熊野寮は日本人男子学部生のみが住める寮でした. しかし現在熊野寮には, 性別も年齢も国籍も問わず, さまざまな寮生がいます. これも, 熊野寮生が「自分たちのことは自分たちで決める」「寮は全京大生に広く開かれるべきだ」といって, 1973年に取り決めたことです.

  このように, あげればきりはないのですが, 熊野寮の「自治」のすばらしさは, 「熊野寮のことについては熊野寮生が取り決め, 運営する」という自治の原則を徹底して貫いていることにあります.

	\subsection{大学自治という基本精神}

	大学自治という言葉をご存知でしょうか. 歴史を振り返ったとき, 戦時に大学は戦争協力をしたり, 国から戦争協力を強要されたりしました. その反省から, 戦後の大学人が不当な支配に屈しない教育の精神や, その証としての, 国に対する大学自治を打ち立てました. 戦後の寮自治も基本精神は同じところから始まります. 戦争協力のような国の不当な圧力に屈しないために, 学問の自由・機会均等を守るために, 寮の自治は行われてきました. 現在でも, 熊野寮が獲得している多くの権利(上記のような入退寮権の獲得や食堂の防衛など) は, 寮生が脈々と自治を行い, 経済的に苦しい学生のための熊野寮を守ってきたからこそ, こうしてあるのです.

	\subsection{自治のすばらしさ}

	さて, それでは自治ということについて, もう少し深く考えていきましょう. 「自治」のすばらしさとはどういう点にあるのでしょうか. それは第一に, 最も熊野寮のことを考えた運営ができるということです. 熊野寮のことは寮生が一番よく知っているのは言うまでもありません. その, 寮生が議論して導き出された運営は最も熊野寮生のことを考えた運営です.

  もちろんそのためには, 寮生一人一人が寮のことに積極的に関わっていかなければ成り立ちません. 毎週行われる専門部会, それぞれの時期に活動する専門委員会, そして, 月に二回のブロック会議と年に二回の寮生大会. これらの寮の諸会議を通して, 熊野寮では寮生同士が意見を活発に交わしています. 従って, これらの会議は, 自治の根幹をなす部分であり, 寮生全員の出席が義務付けられています. 「これだけ会議があると, 大変だ」という声が聞こえてきそうですが, 大したことはありません. なぜなら, 他ならぬ自らのことを決める場なのですから.

  そして第二に, 自分で自らのことを決定できる喜びです. 本来はあたりまえであるはずのことですが, 今の社会ではそれすらも困難になっているのが現状です. そのような中で, 熊野寮では寮生の寮生のための寮生による運営ができるのは何よりも楽しいことです.

  第三は人と人とのつながりです. 例えば寮内で行われる季節ごとのコンパや, 談話室などでの夜を徹しての議論や麻雀に飲み会, 回生・性別・国籍を問わないその人間関係は, 自治を行っている熊野寮だからこそです.

  自分がやりたいことを周りの人に呼びかけて, 企画として実行する. 寮生の想像力に上限がない以上, 自治をしている限り熊野寮の可能性・創造性・発展性は無限大です.

  ところで, 「自分たちのことを自分たちで決める」というと, 個人主義的思考に勘違いされそうですが, そうではありません. 熊野寮は寮生みんなで運営しているものです. そこでは, 最低限のルールというものがあります. それは, 相手のことを考えるということです. 例えば, 熊野寮には当番制の仕事があります. この仕事を誰かがサボったらどうなるでしょう. その分の仕事は別の寮生に負担となります. そのようなことは絶対に認められません.


	\subsection{寮自治の根源}

	最後に, 寮自治の大事な基本精神を.
		 
  自治とは寮生みんなで行うものです. ですので, どのような立場・性別・国籍の人であろうとも, どれだけ利害が対立してようとも, あくまで対等な立場で話し合いを行うことが絶対不可欠です. こうした対等な話し合いで形成された信頼関係がないと, 安心して共同生活なんておくれません. このような信頼関係を築くためにも, 各寮生がお互いのことをよく知り, お互いのことを考え, 立場が異なる人とも対等に接して議論を重ねる努力をしなければいけません. これを, 徹底討論の原則といいます.

  また, 熊野寮では何かトラブルが起きたときも, 押し付けや刑罰でよしとするというのではなく, 当事者が納得できるように, 最大限努力します. 寮生一人一人が, 本当に信頼できる熊野寮を目指しているからです. このような, 信頼関係が熊野寮自治の根源なのです.

  長くなりましたが, これで自治の紹介は終わりです. みなさん納得していただけたでしょうか. まだ「?」というところもあるかもしれませんが, それは入寮されてからということにしておきます. 寮はみんなが作っていくもの. たくさんの信頼できる仲間に囲まれた, 心躍るような寮生活を, 熊野寮でともに過ごそうではないですか. みなさんの入寮を心よりお待ちしております.
