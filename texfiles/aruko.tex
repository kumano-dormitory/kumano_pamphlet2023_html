

\section{歩こう歩こう}

寮に入ると運動不足になる人がかなりの数いると思われる。そういう人の助けになるかは分からないが、入寮した場合のおすすめ散歩スポットを適当に書いていく。 \hfill (文責:胡椒少々)
%	\vskip0.5\baselineskip
	

\begin{multicols}{2}

\subsection{メジャーなスポット}
\noindent 寮生が歩きそうな場所を紹介する。

\subsubsection{鴨川}

\noindent
\scalebox{0.88}{朝★★★★★昼★★★☆☆夜★★★★☆危険度★★☆☆☆}
\vskip0.3\baselineskip
寮から丸太町通りを西の方角へ歩いて五分。京都では珍しい大きめの川がある。といっても大自然とは程遠い、文化と文化の間を流れる川である。南に行かなければ、カップルもたいしていない。晴れた朝は、出はじめの太陽の白い光が、穏やかな、並足のような流れにきらきらと映り、なんとも暢気で、散歩には最適だ。

昼間歩くなら、食べ歩き厳禁。トンビが必ずかっさらっていく。夜は、寮の先輩によれば、大声で歌っても許されるとのこと。夜歌いたいけどカラオケ行くお金ないなというときには、是非行ってみてほしい。ところで筆者はこの前昼の鴨川を大声で歌いながら闊歩したが、特に気にされなかった。川の東側ならどの時間帯でも大抵ほとんど人がいないし、許される。茂みなどもあるので、夜一人で行くなら少し気を付けた方がいいのかもしれない。

また、ダニに噛まれる、ヌートリアが出るなどの情報もある。ある寮生が、日課だと言って、大雨の日にも鴨川沿いをランニングしていたが、鴨川はかなり増水しやすく、危険だ。絶対に真似しないでほしい。

\subsubsection{岡崎公園}
 
\noindent
\scalebox{0.88}{朝★★★★★昼★★★★☆夜★★★★☆危険度★★☆☆☆}
\vskip0.3\baselineskip
広々としていて見晴らしがいい。山の上以外では京都でいちばん開けているのではないかとすら思ってしまう。裏を返せば、夏の昼間なんかはむやみに暑い。熱中症に注意が必要だ。芝生の緑に平安神宮の鳥居の朱が特別に映えるのも、この時季だけれど。 おすすめしたい時間帯は何と言っても朝である。特に、休日の朝。ロビーや食堂にいる暇そうな寮生を誘って、鳥の声を聞きながら、腰かけていいのかいけないのかよく分からないオブジェでのんびりすれば、QOLの向上を感じざるをえない。一人で行っても当然楽しい。すぐ近くに京都市京セラ美術館があり、京都市動物園も遠くない。京大生ならどちらも100円で入れる\footnote{ただし京都京セラ美術館の特別展などは別途入場料が必要}ので、ついでに寄ってみてもいいだろう。

また、夜に五六人で缶蹴りや鬼ごっこを遊びにいくならもってこいだ。ちょうどいい距離にあって、酔い覚めにもなる。



\begin{itembox}[r]{トンビ対策}


	しばしば話題に上がるトンビの対処法をまとめた。わざわざ書くほどでもない気はするが。

	1.両手で食べ物を持たない

	パンを両手で持って食べるなどが一番危ない。逆に、箸で食べるものはあまり攫われないようだ。私もカップラーメンは何回も食べたことがあるが、大丈夫だった。右手がフリーに見えるからなのかしら。分からん。

	2.後ろに障害物のある場所で食べる

	結構なんでもいい。かなり頼りなく見える細っこい木でも案外どうにかなる。トンビは基本背中側から肩をかすめるように急降下して手の中の食べ物を奪っていくので、後ろにちょっとした障害物があれば、急降下の邪魔となって近付けない。
	
	3.警戒する

	単に警戒してトンビを睨みつけるだけでもある程度の効果がある。ただし気は休まらない。


\end{itembox}



\subsubsection{銀閣・大文字山}

\noindent
\scalebox{0.88}{朝★★★★☆昼★★★☆☆夜★★★★★危険度★★★★☆}
\vskip0.3\baselineskip
どうしてなのか、大学生というのは深夜に山に登るのが大好きな生き物らしい。日付が変わるころに赴いて見かけるのは、なにも寮生だけではない。夜景に向かって服を脱ぎたがる迷惑な酒飲み連中も、大声で罵り言葉のしりとりをしている迷惑な酒飲み連中も、みな等しくこの山を愛している。

だが、寮生だけの楽しみ方もある。それは、寮から大文字山、大文字山から寮までの道を楽しむというものだ。これを書いていたとき食堂にいた寮生に、黄金ルートを考えてもらった。まず東大路通を北上して大学へ向かう。夜の大学はわくわくする。昼の大学には行けない寮生たちだというのに......。今出川通に抜け、哲学の道を通って慈照寺(銀閣)に向かう。寺を訪ねるなら、坂に日の当たる昼過ぎがいいとは思うけれど、寺の門のあたりにいるだけでも文化的な気分にはなる。寺から大文字山はすぐだ。山登りを楽しんで、鹿ケ谷通りを歩く途中で銀水湯に浸かって疲れを癒し、今度は白川通をくだっていく。銀水湯はちょっと奥まったところにある銭湯だ。お湯がすべて軟水で、とても気持がいいらしい。丸太町通との交差点にある寮生なじみのガソリンスタンドが見えたら、そのまま西へと寮に帰る。

長い散歩を希望する人は、山を下って滋賀に行ってもよい。ただ、山は危険なので、よく道を知っている人と行くなど、安全には十分な注意が必要だ。


\subsubsection{祇園}
\noindent
\scalebox{0.88}{朝★★☆☆☆昼★★★☆☆夜★★★★☆危険度★★★☆☆}
\vskip0.3\baselineskip
夜、酔っぱらった寮生と一緒に、なんとなく東大路通を南下していると、なぜか決まって祇園に行きつく。八坂神社が広くて楽しい。途中で地下鉄東山駅付近を通るけれども、家や店のかすんだ窓に電灯がところどころついて、それが自然と知恩院の黙へ続いているあの雰囲気もいい。もっと南下すると清水寺に続く坂があって、走り回れる。長生きしたいなら、くれぐれも転ばないように。夏の暑い夜、坂を登り切ったところに立っている涼しげな自販機は格別だ。

\vskip0.3\baselineskip

\subsection{個人的におすすめの場所}

\noindent
あまり寮生が行ったという話を聞かない場所や、散歩する場所としては認識されていないと思われるが、筆者自身が好んで散歩する場所を書く。


\subsubsection{丸太町通を東に行く}
\noindent
\scalebox{0.88}{朝★★★★☆昼★★★★★夜★★★★☆危険度★☆☆☆☆}
\vskip0.3\baselineskip
丸太町通を渡ってどんどん東に行く。丸太町通の東の方は、昼間あかるく、夜はひっそりとしていて、散歩場所にぴったりだ。左手側にうさぎの置物がたくさんある神社や、ガソリンスタンドを横目に、その先まで行くと泉屋博古館がある。これはいい博物館である。古代中国の青銅器がメインなのだが、とにかくコレクションがいい。それから、建物がいい。広さに金の力を感じる。昼間もし開いていたら、覗いてみるのをおすすめする。これは筆者の独断だが、京都の博物館はここと有鄰館が抜けている。

南に行って、南禅寺や永観堂禅林寺を訪れるのもよい。

\subsubsection{丸太町通を西に行く}
\noindent
\scalebox{0.88}{朝★★★☆☆昼★★★★★夜★☆☆☆☆危険度☆☆☆☆☆}
\vskip0.3\baselineskip
丸太町通をどんどん西に行く。寮が京都の東寄りにある関係で、こちらはなかなかに長い道のりである。二条城前の駿台を見て知り合いの顔を思い出したり、円町の駅の自転車置き場の若干寂れた様子に独りよがりな郷愁を覚えるのもここならではである。最後まで行くと、山のふもとの謎の自治体に着く。家々の連なりは古くからの生活を想起させるのに、道がなぜか綺麗に舗装されており、夕方の五時には「遠き山に日は落ちて」の歌が流れる。車止めにもたれてそれを聴き、日々のつくりもののゆとりを思い出すと、満足して寮に帰る。帰り道で、急に分岐が現れるが、のんびりしすぎていると、間違えて北の妙心寺の方まで連れていかれてしまう。

\subsubsection{下鴨神社の東側}
\noindent
\scalebox{0.88}{朝★☆☆☆☆昼★★★★★夜★☆☆☆☆危険度★★★★★}
\vskip0.3\baselineskip
糺の森に行く寮生はよく見るが、そのまま神社の東側の道を北上した話は聞いたことがない。夕方に行くと、本殿の真横あたりを過ぎた途端に鴉の大群が杜の葉をざわめかせ、とても怖い。とても怖いが、神秘的でもある。この帰り道に限って、筆者もオカルト的なものを強く信じたくなってしまう。

\end{multicols}


%\vskip0.5\baselineskip
締切を三週間も延ばしていただき、原稿の遅さワーストスリーにノミネートされてしまったため、またこの恐ろしい遅筆のおかげで、この紹介文を書くのに試験直前の一日を全て使ってしまった。だが、昨日覗いた、世俗と心中している系ジモティー本屋に置かれていた本に、先延ばし癖のある人は才能があると書かれていたので、このかなしき性も誇って生きようと思う。それに、そもそも数日前からの試験勉強などできるわけがないので、なんの問題もない。風呂に入れなかったのは残念だが。お腹が空いた。牛乳でも買って飲もうかね。




