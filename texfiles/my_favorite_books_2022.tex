\section{マイ・フェイバリット・ブックス2022}\label{sec:favorite_books}

(おもむろに流れ出すマイ・フェイバリット・シングスのメロディ)

入寮パンフ誌上特別企画「マイ・フェイバリット・ブックス2022」です。

かねてから、いつか紙数を気にすることなく、好きな本について自由に好きなだけ書きたいと思ってきました。寮の事務室の雑記帳だと、長々と書くことをどうしても遠慮してしまうのです。私の書く字は大きくて、ページを何枚も使ってしまうからです。しかし、昨年の入寮パンフで、『熊辞雲』という熊野寮の用語辞典編纂について、徹夜明けのステーキ丼くらいボリュームのある文章を掲載してもらったとき、私は気づいたのです。編集委員の人々は、パンフが分厚くなればなるほど、うはうはと喜ぶことに。(編集作業ほんとうにおつかれさまです)

そういうわけで、まったく熊野寮とは関係ありませんが、私が去年読んだ小説とその周辺から面白かったものを選んで、読んだ順に好き勝手に紹介していきます。日記をもとに書くので、書誌情報が不完全なことはご容赦ください。

(トーンダウンするマイ・フェイバリット・シングス)
\\
\\
\textbf{『ナイルに死す』アガサ・クリスティー(黒原敏行訳、ハヤカワ文庫):2月(読んだ時期)}

「アガサ・クリスティーってこんなに面白かったっけ?」とびっくりした作品。いや、アガサ・クリスティーは100年前からずっと面白いんだけど。小学生のとき、クリスティーの短編を集めたシリーズ(黄色い装丁のハードカバーのやつ)を、コンプリートしたくてがんばって読んでいたのですが、そのときは図書館の同じミステリ棚にあったホームズや怪人二十面相や明智小五郎のほうがかっこよくて、卵みたいな形の頭で変な口髭を生やしたうぬぼれの強いベルギー人とか、お節介なオールドミスとかには正直あまり魅力を感じなかったんですよね。なんか、「生牡蠣はちゅるんと飲み込むもので噛まないから、ストリキニーネ入りで変な味になっていても気づかない」みたいな推理を読んで、「いや生牡蠣噛むだろ」って驚いたことを一番覚えている。この頃の牡蠣はいまより小さかったのかな……?それともみんな、牡蠣は噛まずに飲み込んでますか?一度、小学生か中学生くらいのときにどうしても気になって、噛まずに飲み込んでみたことがあるのですが、ちゃんと味わわなくてもったいないことをしたな、と残念に思いました。まあとにかく、アガサ・クリスティーを読んだのは本当に小学生以来でした。

この作品は、まずタイトルがかっこいい。あと、ポワロが出てきます(雑な紹介)。殺人事件の舞台となる、ナイル川クルーズ船に乗り合わせたほかの乗客たちの人間模様も面白いです。個人的には、社会主義者のファーガスン青年のその後が気になる。スコットランド語で、「とんでもないくらいの幸福。そのあとには必ず災いが起きる、そんな幸福。こんな幸せはほんとであるはずがない、そういう幸福」を指す「フェイ」という言葉が印象に残っています。翻訳できない世界のことばですな。

読書のお供として合いそうな飲み物を独断と偏見で選べば、とびきり熱くて濃いブラックコーヒーです。
\\
\\
\textbf{『坂の上の雲』司馬遼太郎:2〜3月}

愛媛松山に旅行に行ったとき、町中に『坂の上の雲』ネタが散りばめられていて、読まなきゃなと思ってから約一年。読みはじめる直接的な動機になったのは、ゴールデンカムイにはまったことでした。『坂の上の雲』は、日露戦争とそれに至るまでの近代日本の話です。でも実際に読んでみると、進路の悩みとか、同部屋の親友が突然出ていっちゃうとか、その当時の私自身と重なるところが多かったことから、主人公の一人である秋山弟の少年〜青年期を描く第1巻を一番面白く読みました。全体を通じて、明治の変人奇人が数多く登場します。こういう変人奇人が政治の舞台で活躍できるのは、官僚制がそれほど厳密でない草創期だからこそなのでしょうか。

私が大好きな田中芳樹『銀河英雄伝説』で見覚えのある語句やフレーズがたくさん出てきたところも興味深かったです。田中芳樹、『坂の上の雲』を参考にしたのかな。司馬遼太郎を読んでるときって「次なに読もう?」ってすごくわくわくしてるのに、読み終わったらそのあと一年くらい司馬遼太郎を読まない現象、何なんでしょうね。司馬遼太郎を読むエネルギーが足りていない。そろそろ一年くらいたつし、次は『韃靼疾風録』とか『街道をゆく』を読んでみたいです。

『坂の上の雲』には、日本酒か焼酎かウォッカか中国酒か、度数の高いお酒が合いそうな気がします。でもそうすると文章が追えなくなるので、秋山兄のような酒豪でなければ、素面で読むのがよいでしょう。
\\
\\
\textbf{『深夜特急』沢木耕太郎:5月}

小説じゃないじゃん。小説じゃなかった。あれ、小説じゃないよね?面白くて一気に読みました。毎日が祝祭のような大都市に住む下層の人々の哀しさと、それと表裏をなす美しさが描かれる香港編が好きです。それと、ある画家と彼をめぐる霊感を持った二人の女性の話が、謎めいていて記憶に残っています。

おそらく寮に入ったら(大学に入ったら?)、この本を読んでいる人も周りに多いと思うので、どの場面が印象に残っているかきいてみるのも面白いと思います。

飲み物としてこの本で印象的なのはチャイですが、きっと現地で飲んだ現地のチャイでなければ本物じゃないのでしょうね。
\\
\\
\textbf{『風が強く吹いている』三浦しをん:7月}

ほとんどゼロから箱根駅伝に挑む若者たちの話です。冒頭、主人公の走(カケル)と清瀬が運命的に出会う場面からぐっと心をつかまれてしまいました。「きみしかいない。きみだけなんだ」という熱い思いに弱い。かっこいいドラマチックな文章に胸が熱くなります。自分も全力で走っているような勢いで最後まで読み終えました。

飲み物をのんびり飲んでる暇はない!走りながら水かアクエリで水分補給だ!
\\
\\
\textbf{『秘密機関』アガサ・クリスティー(嵯峨静江訳、ハヤカワ文庫):9月}

こんなに面白い小説があったんですね。若くて陽気で機知に富むトミーとタペンスが、愉快な会話を繰り広げながら危険な冒険に飛び込んでいく物語です。田中芳樹『月蝕島の魔物』のニーダム青年と姪のメイプルのバディが好きな人は、絶対この二人も好きです(響く人が恐ろしく限られたすすめ方)。

とにかくトミーとタペンスがかわいくて最高。推理小説に毒されて探偵に憧れているエレベーターボーイのアルバート少年や、足が必要となったらロールスロイスを即買ってくるとんでもない大金持ちのジュリアス青年など、主人公二人に協力する脇役も面白いです。続巻の『おしどり探偵』では、ハッピーラブラブおもしろカップル(?)になった二人が、いろんな名探偵の真似事をしながら事件を解いていきます。二人の対等な関係が尊い。「物語のハッピーエンドが待ってる場所に帰らないと」というトミーのセリフが好きです。青山剛昌の名探偵図鑑に描かれている二人が見たい!!!!と思って調べたら、描かれていなくてかなしみました。トミーとタペンスのシリーズでは、私はやはりこの二人の相棒関係に萌えるので『秘密機関』が一番好きなのですが、ミステリ好きな人は4作目の『親指のうずき』がお気に召すかもしれません。

読書のお供として合うのはやはり、濃く入れた紅茶でしょうか。あるいはそれとは全然違うけれど、バドワイザーとか、アメリカの爽快なライトビールのイメージです。
\\
\\
\textbf{『終りなき夜に生れつく』アガサ・クリスティー(矢沢聖子訳、ハヤカワ文庫):10月}

こわい。それが読み終わった直後の感想でした。いや、こわいよ。四種の怖さのクワトロフロマージュだよ。

主人公の青年は運命の女性と出会い、ずっと望んでいたとおり理想の家で一緒に暮らしはじめるのですが、そこはジプシーが丘と呼ばれる不吉な土地だったのでした……。不穏なロマンス小説です。あまり好きじゃない遠い親戚みたいな関係の人が自宅にいついてしまって、早く出ていってほしいんだけど追い出せない、みたいな話って不気味ですよね。ほかにもそういうホラーって確かあると思うんですけど、具体的に思い出せません。

この本も多くのクリスティー作品の例に漏れずタイトルが素敵です。恩田陸に同名の作品があって、それがきっかけで膨大なクリスティー文庫の赤い背表紙の中から手に取りました。原題は “Endless Night”。ポワロやミス・マープルの出てこない単発の作品になります。アガサ・クリスティーって基本的にネタバレ厳禁だと思うのですが、ネタバレ部分に語りたい面白さがあるのがつらいところです。この本は、親しい人に読ませて感想を聞きたい。

この本を読むなら、砂糖とミルクでこっくりと甘くした紅茶に、酔って悪夢を見るほどたっぷりのラム酒を垂らして……。
\\
\\
\textbf{『鳥肌が』『野良猫を尊敬した日』『絶叫委員会』など穂村弘:11月}
\begin{itemize}
\item
封を開けたあとの牛乳をいつまでに飲めばいいかわからない(「なるべくお早めに」と書いてあるけれど、いつまでなら飲めるのか?)
\item
水たまりを踏んで靴下がじゅぶじゅぶなのに、靴を脱いで上がらなければならないところに来てしまい、立ちすくむ。スリッパがあって、ああ助かったと思い、罪悪感を抱きながらも、じゅぶじゅぶの靴下のまま足をスリッパに突っ込んでしまう
\item
移動中にどの本を読みたくなるかわからなくて、とうとう決められずに5冊くらい鞄に入れて持ち歩くけれど、結局1冊も読まない
\item
何人かで電車に乗ったとき、どこに、誰の隣に座ればいいかわからずに、うろうろ中腰になっている(これは私)
\end{itemize}

これらに当てはまる人におすすめのエッセイです。においをかいだだけでその牛乳が腐っているかどうか判断できるとか、このことあるを予期して替えの靴下を用意しているとか、文庫本をサッとコートのポケットに入れて家を出るとか、真っ先に座席に座って「隣来なよ!」と言えるとか、そんなスマートなあり方に憧れたことはありませんか?この本がスマートな生き方を教えてくれるわけではないけれど、穂村弘自身のスマートではない日々を読むうちに、自分のことも認められたような気がして、すごくほっとできます。何より、読みながら何度も声を出して笑ってしまいました。

穂村弘は歌人です。私は高校生から短歌を作りはじめたのですが、短歌の作り方みたいな本をこれまで読んだことがありませんでした。しかし、過ぐる年の熊野寮祭の企画で行われたビブリオバトルで、寮生M君が穂村弘の『はじめての短歌』という本で優勝していたことで(私は参加したわけではなく、総括の文章で読んだだけですが)、初めて穂村弘の短歌の本を手に取りました。この『はじめての短歌』という本がべらぼうに面白かったのです。その後、穂村弘のエッセイを次々と手に取るに至り、かっこわるく見苦しく、どこにいてもきまりの悪い思いをしていた私は、とても安らかな気持ちになれたのでした。ありがとう、M君。

「新しい本なんて読む心の余裕がないよー」というときにも、いや、そういうときにこそおすすめです。ぴったりな飲み物は思いつかないので割愛!
\\
\\
\textbf{『赤毛のレドメイン家』アンソニー・フィルポッツ(創元推理文庫):12月}

昔、はやみねかおるのインタビュー記事で紹介されていたおすすめ本(もう1冊は、天童真の『大誘拐』でした)。全部読み終わってから解説をちらっと見たら、江戸川乱歩の選ぶ現代欧米の推理小説10選だと書いてあり、「あ〜、江戸川乱歩こういうの好きそう〜」と思いました。景色を描写する地の文が美麗です。この小説のイメージの飲み物は、ルビーを溶かしたように赤いカンパリソーダ。
\\

ほかに、A・A・ミルン『赤い館の秘密』(くまのプーさんの作者が書いた、殺人込みの推理小説。探偵青年とワトスン役のコンビが無類にかわいい)、レイ・ブラッドベリ『何かが道をやってくる』(文章が詩的)、アイザック・アシモフ『黒後家蜘蛛の会』(会のメンバーたちの毒舌の応酬が楽しい)なども面白かったです。『何かが道をやってくる』は、邪悪なカーニバル、悪夢のサーカス、恐怖のパレードがひたひたと近づいてくる物語で……あ、原稿を催促する編集委員の足音が、ひたひたと真夜中の廊下に聞こえてきました。これは冗談ではありますが冗談ではなくて、今年の編集委員さんはなんとしてでも一次〆切(一次〆切と呼んでいるのは執筆者側だけですが)に原稿を提出させるつもりなのがびしばしと伝わってくるので、そろそろ筆をおかなければなりません。
\\

おわりに。結論として、やっぱりミステリーの女王アガサ・クリスティーは面白いです。さすが100年読み続けられているだけのことはある。2023年もこの金鉱を少しずつ掘りすすめていこうと思います。

面白い本にたくさん出会えた一年でした。最近は新しい映画やドラマ、アニメを見始める気持ちにどうしてもなれず、同じもの(銀英伝)を繰り返し見てばかりいて、「気持ちが年を取ってしまったのかなあ」と危ぶんでいたのですが、新しい本は意外とたくさん読んでいました。今年は人生が変わっちゃうくらいの本に出会えるといいな。本当に出会っちゃったらちょっと怖いですが。

以上、マイ・フェイバリット・ブックス2022でした!
\\
(再び流れ出すマイ・フェイバリット・シングス。徐々に高まり、やがて静かに終わる)
\\
\\
(仏子)