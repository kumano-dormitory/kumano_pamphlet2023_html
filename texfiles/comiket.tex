\section{コミケに行った話}\label{sec:comiket}

\bunsekisha{文責}{オタクくん}

\subsection{はじめに}
  こんにちは。オタクくんです。12月末に行われたコミックマーケット101(c101)に行ってきたので、その報告でも書こうかと思い筆をとりました。

\subsection{動機}
  コミケに行くきっかけは、ひょんなことからサークルでの参加をすることになったからです。おそらくそちらについて説明しているコーナーはないと思うので、ここに書こうかと思いますが、「学寮交流会」というくくりでコミケに同人誌を出そうという話になったのです。学寮交流会とは年に2回全国の学生寮の寮生が集まり、議論し、交流する場です。2022年の夏は東北大学の日就寮へみんなで行きました。楽しかったなあ。年に2回顔を突き合わせるだけでは仕方がないので寮祭に遊びに行ったり、近くを通りがかった際には寄っていくなどしていますし、discord鯖で日常的にメッセージのやり取りもしています。熊野寮に限らず、学生寮に入寮した方はTwitter垢があるので連絡くださいね。

  学寮交流会そのものの説明はこのくらいにして、どういう同人誌を頒布したのかの話をしましょう。頒布したのは「RYOUTONOMY」という全国学寮情報誌です。全国の国立大学生寮データや、各寮の詳しい紹介、その他寄稿文などが載っています。ページ数は怒涛の96頁。受験生でほしい方居ましたら、入寮面接のついでに等言ってくだされば特別にプレゼントしましょう。

\subsection{コミケの前から大変だったねという話}
  私は忙殺されていたためコミットできなかったが、編集の段階が結構大変だったらしい。毎度のごとく集まらない原稿、慣れない編集作業・・・お疲れ様です。印刷して搬入するのもかなーり大変でした。まず、宅配便での申込期限を知らなかった。コミケットアピールを事前に読もう!という話ですね。印刷は熊野寮の輪転機でできるから締め切り的なものを何も考えていなかったのがよくなかった。折り込みをギリギリまで行い、夜行バスに100部ほどの同人誌を持ってご乗車、3列独立でも狭すぎて寝るというより意識を失うという方が正しいような状況でした。次回はよく期限を確認して、より良い手段で運びたいですね。それをコインロッカーにぶち込んで、大みそかまでは用事があったので各地を転々としていました。

\subsection{コミケ1日目}
  一人のオタクとして、下見もかねて参加しました。自分がずっと追っかけていたジャンルの本が買えて大満足。企業ブースも回れて大変良かった。京アニのブースがすごかったです(語彙力)。人の多さを甘く見ていたので、酸欠になるかと思いました。りんかい線の運賃高いですね。

\subsection{コミケ2日目}
  5時過ぎに治安が終わっているカプセルホテルを出発し、コインロッカーのある駅で共に出展する者たちと集合。朝7時過ぎにビッグサイトに到着。搬入し、設営作業している間に周囲のサークルも徐々に設営が進み、9時半ごろにはほとんどのサークルが設営完了し、クソデカホールに様々なジャンルのブースが出展しているのは圧巻の一言でした。所謂壁サークルの同人誌でどうしても欲しいのがあったので、番を他の人に任せて入手しに行きました。ゲットした同人誌の話もしようかと思いましたが、キモいし本筋からもそれるのでやめておきましょう。時間があればオタク全振りの記事も書きます。嵐月、しろし両氏の名前だけ覚えて貰えれば幸いです。
  
  「RYOUTONOMY」の頒布状況も順調で、初出展、コネクションほぼなしの30部頒布は誇ってもいい数字だと思っています。次回以降もサークル参加するモチベになりました。真面目な話題だけで半年に一回出すのも大変なので、夏号の寄稿文はみんなに思い思いのものを書いてもらおうかな、とも思っています。ともあれ、それなりに首尾よく諸々進んでよかったです。関わった皆さん、お疲れ様でした。

