\section{大学当局との確約・団体交渉}\label{sec:kakuyaku}
	熊野寮自治会が大学当局と結んでいる確約と団体交渉について説明します.

	\begin{shadebox}
	\textbf{言葉の解説:大学当局} \\
	「京都大学」には学生個人や多くの学部・寮自治会などの組織も含まれます. それら学生側と区別して大学運営の中枢である事務本部や各担当部署を指す言葉です. 「我々大学当局としては...」というように, 京都大学では職員や教員にも一般的に使われている言葉です.
	\end{shadebox}

		\subsection{確約}
		確約とは熊野寮自治会と京都大学当局との間に取り交わされた約束です. 実際に居住している当事者である寮生の意見を無視して当局が一方的な決定を下すことを防ぎ, 当局と寮自治会が対等に協議できるようにするために, 当局側の権力行使に制限をかけるものです.

		\subsection{「対等」な話し合い}
 		前項で「当局と寮自治会が対等に協議できるように」と述べましたが, 法的に管理権を有する当局と学生の間には大きな権力差があるということが前提になっています. だからこそ, その力の差を可能な限り埋め合わせ, 可能な限り対等な話し合いによって学生のよりよい福利厚生を実現しようとしているのです.

		例えば, 寮生は交渉で発言する際にも当局に対して名乗ることはありません. 所属を特定されて個人単位で弾圧される危険性があるからです. 京大では近年もアカデミックハラスメントが起きています. 自分の「家」について当局と意見が違っただけで, 学業面で不当な扱いを受けることがあってはなりません.もちろん, そもそも組織間の交渉なので, 寮自治会の立場を述べるのであれば, 個人が名乗る必要がないということでもあります.

		\subsection{確約の引き継ぎ}
		確約書は厚生担当副学長の署名によって結ばれており, 形式的には副学長個人の寮に対する約束となっています. しかし, 「熊野寮自治会と京都大学の基本確約」(\pageref{page:確約}頁参照)とあるように, 本来的に確約は組織間の約束です. 例えば担当者(任期2年の副学長)が交代したからといって, 大学と自治会の約束がいちいち白紙に戻っては困ります. その為に, 副学長が交代しても引継ぎが自動的に, 確実になされるよう, 「次期以降の学生担当理事, 厚生補導担当副学長に引継ぐ. 」という条文が定められています.

		ちなみに現在は, 当局の責任者として平島崇男副学長が寮自治会と確約を結んでいます.

		\subsection{団体交渉}
		当事者(寮生は勿論, 今後入寮しうる京大生, その他関係者など誰でも) がすべて自由に参加できる, 公開の交渉形態です. 確約の引継ぎや, 新寮の建設など寮にとって重大な事案についての団体交渉が, 学内の大教室や寮食堂などで開かれてきました.

		例えば対照的に, 少人数の交渉形態, つまり寮自治会の代表者数名だけが参加できる協議の場というのはどうでしょうか. 経済的に苦しくてアルバイトで忙しいため自治会の交渉事に中心的に参加できていない者や, 入寮したばかりで確約などの詳しい知識が無かったりする者など, 「自分の家のことが協議される場に参加したい」という思いは皆同じです. すべての当事者(特に居住している寮生) には協議・意思決定の場に参加する権利があると考え, この形態で交渉することが条文でも定められています. 団体交渉という形態を拒否すること自体も確約違反なのです.

		\label{page:確約}

		\begin{tcolorbox}[colback=white, colbacktitle=gray!30!white, coltitle=black, title=熊野寮自治会と京都大学の基本確約,breakable]

		熊野寮自治会と京都大学は, 2004年3月31日に結ばれた, 両者の確約書に則り, 以下の内容に合意する. 両者は熊野寮が京都大学学生全体に開かれたものであることを確認し, その一層の充実のため誠実にこの確約内容を履行するものとする. 両者は学生の居住する熊野寮の運営に関しては, 当事者であり主体的にその責任を果たす学生の自治によることが最良であることを確認し, この認識を基礎にこの確約を結ぶものである.本確約は文書としては二通作成し, 両者が一部ずつ所持するものとする.

		\begin{enumerate}
		\renewcommand{\labelenumi}{(\arabic{enumi})}
		\item 京都大学は福利厚生施設としての熊野寮を設置し, その施設を維持・管理する. 京都大学は熊野寮の一層の充実に努めるものとする.
		\item 京都大学は, 熊野寮の日常的運営が熊野寮自治会によることを確認し, 熊野寮自治会はその運営を誠実に行うよう努力する.
		\item 京都大学は, 熊野寮の改廃や新寮および新規寮建設, 熊野寮に関わる人員配置, またその他熊野寮に重大な影響を与え得る事案に関しては, 公開の場で熊野寮自治会と団体交渉を行い, 合意の上決定する. 加えて, 熊野寮自治会と京都大学は, 一方が提示した議題に真摯に取り組むものとする.
		\item 熊野寮自治会または京都大学は, 両者の協議の場において, 何らかの条件を付そうとする場合には, 相手側の同意を得るものとする.
		\item 熊野寮自治会または京都大学は, 熊野寮に重大な影響を与えうる事案について何らかの計画を構想した段階で, 相手方にその内容を報告するものとする.
		\end{enumerate}

		\flushright{京都大学副学長 \\ 熊野寮自治会}

		\end{tcolorbox}
		

		\begin{tcolorbox}[colback=white, colbacktitle=gray!30!white, coltitle=black, title=確約,breakable]

		%\centerline{\bf 確約}

		上記の「熊野寮自治会と京都大学の基本確約」ならびに2010年12月17日の赤松副学長(当時) による「確約書」を基礎にして, 以下の内容を遵守する.

		\begin{enumerate}
 		\renewcommand{\labelenumi}{(\Alph{enumi})}
		\item 熊野寮食堂の機能の維持・向上について
			\begin{enumerate}
			\renewcommand{\labelenumii}{(\arabic{enumii})}
			\item 過去に熊野寮食堂の食堂労働者を削減した事実を認める.
			\item 現在の熊野寮食堂の食堂労働者の置かれている労働環境が劣悪であることを認め, その改善に努める.
			\item 熊野寮食堂において, 食中毒が発生したり, 感染症が持ち込まれたりしても, これを理由とした食堂廃止は行わない. また保健所の指摘を理由とした食堂廃止も行わない.
			\end{enumerate}
		\item 京都大学熊野寮食堂運営会( 以下, 「食堂運営会」とする) について
			\begin{enumerate}
			\renewcommand{\labelenumii}{(\arabic{enumii})}
			\item 2013年6月21日に改正された「京都大学熊野寮食堂運営会会則」ならびに2014年6月20日に改正された「京都大学熊野寮食堂運営会就業規則」を遵守する.
			\item 食堂運営会雇用調理員に対する一定の雇用責任を認め, 労働災害発生時の補償責任を負う.
			\end{enumerate}
\item 寮内労働者について
			\begin{enumerate}
			\renewcommand{\labelenumii}{(\arabic{enumii})}
			\item 寮内労働者の雇用形態について本来ならば全員大学雇いが望ましいことを認め, その労働環境・労働条件に関しては改善に努める.
			\item 恒常的業務に従事する寮内労働者が退職となった場合には, 後任を補充する. この後任補充の際, 雇用形態や労働条件などの, 労働に関わるすべての条件を, 前任者と同等もしくはそれ以上で確保する.
			\end{enumerate}
		\item 寮の生活環境の向上について
			\begin{enumerate}
			\renewcommand{\labelenumii}{(\arabic{enumii})}
			\item 寮自治会から生活環境に関する要求があった場合には, その要求された箇所について調査を行い, そのための工事や改修について検討する. また, その検討結果を速やかに寮自治会に報告する.
			\end{enumerate}
		\item 熊野寮に対する家宅捜索などについて
			\begin{enumerate}
			\renewcommand{\labelenumii}{(\arabic{enumii})}
			\item 熊野寮に対する家宅捜索の立会いの方法について, 熊野寮自治会からの要求があった場合には, 熊野寮自治会と協議する.
			\item 熊野寮に対する家宅捜索において, 寮自治会への令状不提示, 過剰警備(玄関前等の占拠) , 抗議する寮生や掲示物のビデオ・写真撮影など自治や人権を侵害する行為が行われた場合には, その場で抗議する.
			\item 熊野寮に対する家宅捜索が不当であるかどうかを検討し, 寮自治会に対して, その検討内容を明らかにする. 不当であると判断した場合には, 速やかに抗議する.
			\item 熊野寮自治会から「外部団体により不当な扱いを受けたので抗議してほしい」という要求があった場合, 大学職員がその不当な扱いを現認したか否かにかかわらず, 抗議の是非を検討する. また熊野寮自治会からの要求があった場合には, その不当な扱いについて, 熊野寮自治会と協議する.
			\end{enumerate}
		\item 桂キャンパスの利用について
			\begin{enumerate}
			\renewcommand{\labelenumii}{(\arabic{enumii})}
			\item 熊野寮生をはじめとする学生の桂キャンパスへの通学において, 不便とならないように努める.
			\item 桂キャンパス周辺に新規寮を建設することを検討する.
			\end{enumerate}
		\item 寮自治会と国立大学法人京都大学の関係性について
			\begin{enumerate}
			\renewcommand{\labelenumii}{(\arabic{enumii})}
			\item 中期計画, 年度計画を文部科学省に提出する前に, 寮に対してどのような影響があるのかを寮自治会に対して提示・説明し, 寮自治会からの要求があった場合には, 内容を変更することが可能な協議の場を持つ.
			\item 国立大学法人運営費交付金など大学法人の収入減少を理由とする寮関係予算の削減を行う場合, その重大性などに関する寮自治会との真摯な議論を経て, 寮自治会と合意するものとする.
			\end{enumerate}
		\item 団交--確約体制ならびに確約の引継ぎについて
			\begin{enumerate}
			\renewcommand{\labelenumii}{(\arabic{enumii})}
			\item 学生担当理事, 厚生補導担当副学長は, 熊野寮自治会との間において, 団交--確約体制を維持する.
			\item 学生担当理事, 厚生補導担当副学長は, 「熊野寮自治会と京都大学の基本確約」ならびに本確約を, 次期以降の学生担当理事, 厚生補導担当副学長に引継ぐ.
			\end{enumerate}
		\end{enumerate}

		\end{tcolorbox}