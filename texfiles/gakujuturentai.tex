\section{学術連帯PTの設立について}
\label{sec:gakujutu}

皆様こんにちは。Kyoto Scienceです。このパンフレットを読んでいるのはどういう人がおおいのかな?高校生??受験生??大学生??それとも親御さん???

もしかして…大学院生だったりしませんか??

\hanten{実は京都大学の学生のうち、40\% 以上は大学院生}なんです。

ちょっと説明すると、大学院生は大学を卒業したあと、大学院というところに入って、主に研究を行っている学生です(法科大学院など専門知識習得に特化した大学院もある)。日々\emphbf{高度な学問を学んだり、世界最先端の知を創生する研究活動を行う、大学の主力部隊}です。

京都大学にはいろんな大学院生がいます。学部からストレートで大学院に進学した人、飛び級で大学院に入った人。地方から京都に出てきて研究をスタートさせた人、脱サラして博士号取得を目指している人。ミュージシャンだったけど事故の後遺症で音楽ができなくなってしまって学問の道を選んだ人。企業から単身赴任で派遣されている人。私が直にであった人だけでもこんなにいっぱいいます。専門分野も風俗から微分幾何学まで様々です。

\emphbf{熊野寮は、全ての大学院生を含む、京都大学を舞台に学ぶすべての人に門戸を開いてきました}。その事情は一切問うていません。経済的理由による優先的選考を除き、全て公平な抽選で新入寮生を選んでいます。一切の差別をおこなっていません。

しかしながら、京都大学に占める大学院生の割合が高いにも関わらず、歴史的にも実務的にも大学院生が集う一大拠点であるにも関わらず、熊野寮における大学院生向けコミュニティはほとんど存在していませんでした。全学にもあまりありません。

2022年冬、私を含む有志が\hanten{「学術連帯PT」を設立}しました。\sshatai{\emphbf{学術連帯PTは、大学院生を含む学問を志す学生を}}\\
\sshatai{\emphbf{組織し、連帯を作ると共に、全学の大学院生と熊野寮の連帯を志し、加えてその学術的能力を寮自治に生かし自治を}}\\
\sshatai{\emphbf{活性化させることを目的として、熊野寮に設置されたプロジェクトチーム (PT) です}}。
%斜体は自動では折り返せないので、人工的にやっている。

当面は院生新歓をはじめとする緩やかな連帯を築くと共に、語学・数理能力・抽象的思考能力などを寮自治に生かす活動を行っていきます。また学会・論文誌をはじめとする学術界の関係各所と連帯し、熊野寮のプレゼンスを高めつつ、大学院生の福利厚生や生活向上に取り組みます。同時に研究室アルバイトの情報や学振をはじめとする各種フェローシップの情報などを収集・蓄積すると共に、大学院生と企業をつないで社会貢献・報酬を得ることも視野に入れています。

PTへの入会資格は大学院生に限っていません。大学院生のサポートをすることも業務の一部ですし、京都大学には学部生から研究所にこっそり入り浸って研究する人も多いからです。今のところ誰でも入れる形にしています。\emphbf{ご興味を持たれたそこのあなた。まずは春の院生新歓に来てみませんか。連帯していきましょう}。

\hfill 2022年大晦日。新年の訪れと迫りくる締切に怯えながら。\qquad 初代PT長Kyoto Science 
