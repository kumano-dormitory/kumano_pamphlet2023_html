\section{募集要項} \label{sec:admission}

\komoku{募集対象}
京都大学の学生、大学院生、研究生、その他本学に学籍のある者(科目等履修生、聴講生など)を対象とする。性別、国籍は問わない。

\komoku{募集人数}
現在の空き人数(100名程度)を募集する。

\komoku{選考}
入寮希望者が募集人数を超えた場合には選考を行う。

\komoku{入寮基準}
寮自治活動を理解し、積極的に参加すること。

\komoku{選考方法}
部屋割りの都合上、選考は男女に分かれて行われる。また下記のように経済的に困窮している者および留学生には優先的に選考を行う。経済選考、留学生選考に応募した者は、経済選考・留学生選考に漏れた時点で自動的に一般選考にまわる。


% Please add the following required packages to your document preamble:
% \usepackage{multirow}
\begin{table}[hbt]
  \begin{tabular}{|l|lp{40zw}|}
  \hline
  一般選考 & \multicolumn{2}{l|}{全員に対して抽選を行って選考する。} \\ \hline
  \multirow{3}{*}{特別選考} & \multicolumn{2}{l|}{一般選考に優先して行われる。} \\ \cline{2-3} 
   & 経済選考 & 経済的に困窮している者は、経済選考に出願できる。出願希望者は、出願参考書類の所定の欄に記入し、既定の書類(後述)を添えて提出すること。困窮状況が明らかな場合は優先的に入寮を認める。\\ \cline{2-3} 
   & 留学生選考 & 留学生は経済選考で必要な証明書を提出するのが困難であること、また留学生が置かれている社会的状況を考慮し、留学生枠を設定する。経済的に困窮している留学生の入寮希望者は、留学生枠への応募が可能である。その際、留学生選考用書類も併せて提出する必要がある。応募者の総数が留学生枠を超えた場合は、留学生枠内で抽選を行う。\\ \hline
  \end{tabular}
  \end{table}

\komoku{出願書類}
下記の書類を面接時に提出すること(原則面接日に提出すること。やむを得ず当日用意できない事情がある場合は、面接受付時にその旨を申し出たうえ、入寮手続きまでに必ず提出すること。)。なお、一度提出された書類はいかなる理由があろうと返却しない。
\begin{table}[htb]
  \begin{tabular}{|l|p{43zw}|}
    \hline
    全員必要なもの & \vspace{-5mm}\begin{itemize}
        \item 学籍を確認できるもの(学生証・合格証明書)の写し。*受験生は面接時に受験票を持参し、入寮手続き時に合格証明書の写し等を提出して下さい。
        \item 入寮願(署名の上、顔写真を貼付)
        \item 同一世帯全員の住民票写し(発行から3ヶ月以内、マイナンバー記載の\textbf{ない}もの。役所窓口で「世帯全員分」と言えば貰えます。留学生は提出不要)
        \item 出願参考書類
    \end{itemize}\vspace{-5mm} \\ \hline
    経済選考 & 上記に加えて以下の書類を提出して下さい。
    \vspace{-3mm}
    \begin{itemize}
      \item 出願参考書類の所定の欄に経済状況(家族の人数、収入の有無、就学の状況など)についての説明をできるだけ詳細に記入すること。十分な情報が記されていない場合、経済選考に出願できない場合がある。
      \item 家族全員の所得(または家庭の経済状況)を示すことのできる公的機関の発行する証明書あるいはその写し (例:源泉徴収票、確定申告書、失業保険の給付証明書、住民税の免除証明書など。)
      \item 特別な事情がある場合には、そのことを詳しく書いた書類やそれを証明できる公的な書類
    \end{itemize}\vspace{-0.8cm} \\ \hline
    留学生選考 & 全員必要な書類(住民票を除く)に加えて留学生選考書類を提出
    \\ \hline
  \end{tabular}
\end{table}


\noindent 入寮願、出願参考書類、留学生選考書類は熊野寮HPの「入寮希望者の方へ」ページ下部から入手するか、このパンフレットに挟み込まれているものを使用して下さい。

\newpage

\komoku{面接と見学}
  \begin{itemize}
		\item 入寮を希望する者は、必ず入寮面接を受けなければならない。
    \item この面接は入寮希望者が予め寮自治について理解するために行われる。
    \item 面接の場に入寮希望者本人以外は同席できない(付添人には面接中の待合スペースを用意している)。
    \item 面接による選考は、あまりに寮自治への理解がないと判断される場合を除いて行わない。
    \item 面接は\emphbf{熊野寮食堂にて}以下の日程で行われる。\emphbf{予約等は必要ない}ので都合の良い時間に来ること。\\
    (2月の日程と3月の日程で受付終了時間が異なるので注意)
      \begin{itemize}
        \item 2月25日(土)、26日(日):午前10時から正午、および午後1時から午後7時
        \item 3月10日(金)、11日(土)、12日(日):午前10時から正午、および午後1時から午後5時
      \end{itemize}
    \item 面接の所要時間は1〜2時間程度。また面接後に寮内の見学をすることができる(付添人の同伴可能)。
	  \item \emphbf{京都大学を受験しているものは合否が判明してなくとも面接を受けることができる}。ただし不合格であった者の入寮は認められない。
    \end{itemize}
	
%\newpage % これはレイアウトの都合なのでいらなさそうなら消す
		
\komoku{選考後の流れ}
	\begin{enumerate}
		\item 当落発表

		3月13日(月)夜に選考を行い、翌14日(火)に全ての出願者に選考結果を電話で通知する。
		\item 繰り上げ当選

 		落選者は、キャンセル待ちに登録することができる。選考結果の連絡を受けた際に申し込むこと。

 		選考当選者のキャンセルが発生するたびに、キャンセル待ちに登録した者の中から、繰り上げで追加の当選者が確定していく。追加で当選した者には、当選した時点で電話で連絡する。
	\end{enumerate}

\komoku{入寮手続き}
		期日までに行われなければ入寮資格を失う。やむを得ず面接日に不足書類があった者は手続きまでに必ず書類を持参すること。
		
		例年、入寮手続き最終日を中心に手続き希望者が殺到して長蛇の列ができ、事務員さんにも負担がかかっています。またコロナ感染リスクを考慮し、新入寮生の皆さんには、以下の3点を注意していただくようお願いしたいです。
		\begin{itemize}
		    \item 入寮手続きの際はマスクの着用と手指のアルコール消毒をお願いします。(窓口に設置予定)
		    \item \emphbf{必ずしも居住開始日に手続きする必要はありません}。入居日から手続き締め切り日までのうちで出来るだけ窓口が空いている時間帯を選んで手続きに来てください。また、とくに最終日は混むので出来るだけその前までに済ませてください。
		    \item 入寮手続きの際に保護者が同伴するのは、出来るだけ遠慮していただくようお願いします。
		\end{itemize}
        
\komoku{受付日}  3月に以下の日程で受け付けている。%場所はかかんでええのかね
\begin{table}[h]
  \begin{tabular}{|l|c|c|c|}
    \hline
    受付日 & 曜日  & \multicolumn{2}{|c|}{受付時間}                  \\ \hline \hline
    23日 & 木 & 9:45〜12:45 & 14:00〜15:30 \\ \hline
    24日 & 金 & 9:45〜12:45 & 14:00〜15:30 \\ \hline
    27日 & 月 & 9:45〜12:45 & $ \times $   \\ \hline
    28日 & 火 & 9:45〜12:45 & 14:00〜15:30 \\ \hline
    29日 & 水 & 9:45〜12:45 & $ \times $  \\
     \hline
  \end{tabular}
\end{table}

\komoku{入寮オリエンテーション}
	3月25日(土)14時からの入寮オリエンテーションに参加すること。また、これに不参加の者は強制退寮となる場合があるので、必ず参加のこと。(所要時間:4〜5時間程度)