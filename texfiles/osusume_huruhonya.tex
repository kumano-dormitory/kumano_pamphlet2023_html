\section{文学部5回生がおすすめする古本屋・図書館・博物館}\label{sec:huruhonya}
\bunsekisha{文責}{仏子}

\subsection{古本屋編}
\subsubsection{エルカミノ El camino}
出町柳駅から歩いていける距離、出町桝形商店街にあるおしゃれな古書店。セレクトショップっぽい感じ。岩波文庫などのほか、美術書・建築書・映画のパンフレットなどを多く取りそろえている。値段はかなりリーズナブルだと思う。『ペルシアの遺宝』(写真・並河萬里 解説・林良一、新人物往来社)という全3巻のハードカバー函入り写真集を300円+税で手に入れたことがある。ほかの掘り出し物は、NHKの『写真集シルクロード』や『砂漠の華 トルコタイルの花唐草』(写真・並河萬里、文化出版局)など。

\subsubsection{ふるほん上海ラヂオ}
この店こんな名前だったんだ、とこの原稿を書くために調べて初めて知った。店の前に大きく掲げられている「古本市」が店名かと思った。同じく出町桝形商店街にある、さらに価格帯が安い印象のお店。特に文庫本は、同じ本を買うにしてもブックオフより安く手に入る可能性がある。エルカミノで『シルクロードの心 平山郁夫展』という美術展の図録を買ってからこちらのお店に行ったら、より安く売られておりショックを受けた記憶がある。前述のエルカミノより奥まった位置にあるが、こちらを先に覗いたほうがいいかも。美術展の図録・絵本・文庫本などを多く扱っている感じがする。ここでの過去の掘り出し物は、エロール・ル・カインの絵本『いばらひめ』(ほるぷ出版)。すぐ近くに、熊野寮の厨房が野菜を卸してもらっている井上果物店があるので、あわせて立ち寄ってみることをおすすめする。出町桝形商店街にはもう一件、名高い映画館・出町座のなかの古書店があるが、おしゃれすぎて足を踏み入れたことはない。

\subsubsection{吉岡書店}
別に私がここで紹介しなくても、入学すればおのずと知ることになる京大直近の古書店。百万遍の交差点北東角にあり、店の前にも本が並べてあるのですぐにわかる。岩波新書・岩波文庫・講談社文庫・中公新書や、各分野の専門書はまずここへ。一度行くと必ず数千円単位の買い物をしてしまう魔窟。名前が平凡すぎて覚えられたためしがない。

\subsubsection{ブックオフ三条駅ビル店}
寮から一番近いブックオフ。バイトの連勤を乗り切るために、出勤前にここで『銀魂』を5冊ずつ買って、退勤後に夜明け近くまで読みふける日々を繰り返した2回生の夏休みが懐かしい。回転が速いので、気になる本があったらその場で買ったほうがいい。大学生が売りに来るからか、結構岩波文庫の類や教科書、授業で紹介される本も置いている。掘り出し物は、竹本健治『匣の中の失楽』や、エロール・ル・カイン『アラジンと魔法のランプ』など。おそらく河原町オーパ店のほうが品揃えがいいのだろうが、河原町オーパ店は遠い。

\subsubsection{春の古書大即売会}
京都古書研究会主催の、みやこめっせ(寮の近く)で開かれる大規模な古本市。ゴールデンウィークあたりに開催される。見渡すかぎり本棚が林立する夢のような会場内を一巡してレジに並んだ頃には、買い物かごが腕の筋肉を破壊しそうなほど重い。一度ここでバイトをしたことがあるが、レジ担当として無能すぎて2日目からは後方に回された。でも、バイトを使う立場の古書店主の方々も周りのバイト仲間たちも優しい。森見登美彦『四畳半神話体系』に登場する、夏の下鴨納涼古本まつりにも憧れていたが、京都のお盆は信州人には暑すぎて、5年間で一度も行ったことがない。古本好きの筆者を退散させるくらいだから、京都の夏はよっぽどだと覚悟したほうがよい。蓋をした鍋の中にいるみたいな酷暑です。

\subsubsection{秋の古本まつり}
同じく京都古書研究会主催の、百万遍知恩寺で開催される古本市。
時期は10月末から11月初頭。京大本部キャンパスから北の通りを挟んですぐ。ここでの掘り出し物は、清朝末期の北京を写した写真集2冊と、澤田瑞穂『中国の呪法』(平河出版社)など。1回生のとき、オークションで平山郁夫全集が1000円から出品されていたのに競り落としに行かなかったことを、いまでも悔やんでおり、全集本コーナーは必ずじっくり見てしまう。

\subsection{図書館編}

\subsubsection{京都府立図書館}
寮から歩いて行ける距離。平安神宮にもほど近い文教地区・岡崎に所在する。専門的な本に関してはおおかた大学の図書館で事足りる京大生にとって、公立の図書館はおもに小説を求めて行く場所だと思うが、残念ながらこの図書館ではほとんどの小説が閉架式書庫に納められている。借りたいタイトル・作者を決め打ちしているのであればおすすめ。また、短歌が好きな人であれば、現代の歌集も結構収蔵しているのでおすすめ(筆者の趣味は短歌)。左京区図書館は開架が充実しているが、熊野寮や京大のある左京区が南北に長すぎるために、左京区図書館も左京区役所も左京郵便局も遠すぎるんだよな。知ってますか? 左京区は福井県のめちゃくちゃ近くまでのびているんですよ。寮の近くにあるのは権力(警察署)だけ……。京都府立図書館と京都市立図書館は別々に貸出カードを作る必要があるが、市立図書館のカードを作ってしまえば、ほかのいろいろな区でも共通して同じカードを使うことができる。筆者はバイト先が醍醐という場所なので(豊臣秀吉の醍醐の花見で有名な醍醐寺のあるところ)、よく醍醐中央図書館を利用していた。

\subsubsection{文学部・文学研究科図書館(京大)}
専門的な研究書や原書を納める図書館。京大に存在する二十余りの図書館・図書室の中で随一の蔵書数を誇る。地下書庫が迷宮みたいでわくわくする。閲覧室がわりと空いている。

京大の図書館・図書室をコンプリートしてみたかった。基本的には自分が所属する学部以外の図書館でも本を借りることができるので、これから入学される皆さんにはいろいろな図書館を探検してみてほしい。雑誌の書庫のある文学部東館も、手頃な探検におすすめ。

\subsection{博物館編}
\subsubsection{奈良国立博物館}
通称・奈良博。キャンパスメンバーズの割引が効くので、学生証を提示するだけで、正倉院展に400円で入場できる。しかも常設のなら仏像館は無料。キャンパスメンバーズ(博物館・美術館・動物園などが、学生証の提示によって割引価格で利用できる制度。施設によっては事前登録が必要)であれば、ポイントカードにスタンプを3回押してもらうごとに景品ももらえる。次回の正倉院展の開催形式はわからないが、もし日時指定制であれば、予約した入館開始可能時刻より20~30分くらい遅れていくことをおすすめする。入場前に並ばなくていいし、展示品をストレスなく最前列から見ることができる。

正倉院展以外の企画展も豪華。現存する曜変天目茶碗を3点すべて集めて公開したこともある。しかし、筆者が一番好きなのは、常設のなら仏像館だ。仏像と中国古代青銅器しかないが、筆者は仏像と中国古代青銅器が大好きなので問題ない。そんなに人がいないので、最高の環境でゆったりと仏像を鑑賞することができる。大きな仏像たちももちろんすばらしいのだが、筆者の一押しは仏像の残欠群。ミロのヴィーナスでいえば、失われた腕のほうだけが集まっている。何らかの理由で欠けてしまった仏像の腕や足や頭や瓔珞や天衣に物語を感じる。でも白状すれば、奈良博に行くときの大きな楽しみは、お土産に買った柿の葉寿司を食べながら、寮で日本酒を飲むことです。京都国立博物館より奈良博に行った回数のほうが多い。わたしは、奈良派。

\subsubsection{泉屋博古館}
東山・鹿ヶ谷にある東洋史関連の博物館。中国古代の青銅器が多数展示されている。静かでいい感じ(雑)。休館している期間もあるので、事前にHPで開館時期を確認してから行ったほうがいい。

\subsubsection{河井寛次郎記念館}
間口は狭いけれど、奥に驚くほど敷地が広がっている古民家。床や柱や階段の木の質感とか、陶器の独特のあたたかみとか、調度の趣味のよさとか、奥に登り窯があるびっくり感とか、すべてが楽しく居心地がよい。文学部・高木先生の日本史特殊講義の遠足で連れていってもらった。ブラタモリみたいで楽しい遠足だった。

ここで飼われているねこがかわいいんだよな。まだいるかな。一度そのかわいさを知ってしまうと、人間ってどうしてこんなに猫の前でだらしなくなってしまうんでしょうか。実家が猫を飼いはじめるまでなんとも思ってなかったのにな。逢いみてののちの心に比ぶれば昔はものを思わざりけり。猫とふれあっていると、猫好きになっちゃう寄生虫が脳に棲みつくってほんとうなんですか? 猫の前でよだれをたらしている人に共感できないでいるそこのあなた。あなたはまだ寄生虫に取りつかれていない、さいわいな人です。あー、またあのねこちゃんにあいにいきたいなあ。

\subsection{番外・花見編}
\subsubsection{冷泉通り、琵琶湖疎水沿い}
熊野寮の裏の疎水べり。碧くうつくしい水に、しだれかかる満開の桜が映える。赤煉瓦の水門との取り合わせにも見とれる。あまり観光客がいない。

\subsubsection{醍醐寺}
寮からのアクセスは、京阪・神宮丸太町駅あるいは徒歩で京都市営地下鉄・三条駅へ、それから東西線で醍醐駅まで20分弱。有料エリアを全部回ろうとすると確か2000円くらいかかるが、外から眺めるだけでも咲き満ちる桜がすばらしい。観光客はめちゃくちゃいる。「これが醍醐の花見か~」とつぶやきながらうろうろするだけで楽しい。蝉がしわしわ鳴いている夏もしみじみとしていて好き。実は世界遺産でもある。日本に、京都があってよかった。

\subsection*{}

総括。こういう情報って、人から教えられるより自分でたくさん無駄足を踏んで発見したほうがうれしいですよね。旅や気ままな散歩や余暇の過ごし方まで、他人の効率至上主義におかされたくはない。この記事を読んでいらいらとそう思っている方もおられるはずです。本なり古本屋なり自分が見つけた何かを人におすすめしたいとき、その一念が私の筆を重たくしてしまう。だからこの記事も余計なことをしてるなとは思ったのですが、したり顔で情報通を気取る誘惑に勝てなかった。七つの大罪の一つですね。でも私の短足で歩き回れた京都などごくごく狭い範囲だけですから、皆さんが無駄足を踏む余地はまだ十二分にあります。あなたの京都を発見してください。


