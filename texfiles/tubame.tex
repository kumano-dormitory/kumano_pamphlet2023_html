\section{燕の火の用心}\label{sec:tubame}
\bunsekisha{文責}{番茶}


  その燕の性別を僕は知らないし、燕の性別はこの物語に関係がない。昨年の冬、ひどい寝坊をした燕は南に渡る仲間から置いてけぼりを食らった。生来の方向音痴の燕は、今年は南の国に帰ることはできないと観念して京都で冬を越そうと決めた。大きな橋の下で一夜を過ごし、別の大きな橋の下で次の夜を過ごし、三日目には大きな森で雨に降られた。四日目に屋根を求めて丸太町通をぶらぶらして熊野寮を見つけたようだ。燕は初めの十日間A棟で過ごし、B棟には三日しかおらず、最終的にC4の踊り場に落ち着いた。
  
  朝起きると、燕は自慢の二股の尾っぽをふりふりしてから、空に飛び立つ。夷川発電所のダムの上で三回円を描いた後、AB棟間中庭で手頃な餌を探す。初めはまだみみずや羽虫がいたけれど、十二月初頭に寮生が焼き畑をやって虫が全くいなくなった。そこで燕が目をつけたのがシャワー室裏給湯器の下だ。そこは排熱のお蔭で暖かく、ごきぶりが大変たくさんいた。燕はもう飢える心配はなかった。

  昼は下駄を履いたコシケイが、竹でできた虫捕り網をぶんぶん振り回し、廊下から廊下、階段から階段へと燕を追い回す。からんからん奇麗な音がコンクリの建物に響く。燕が開いている扉を潜り抜け、下駄の音に起こされた不機嫌な寮生をぐるりと旋回すると、コシケイは決まって「おっとっと」と足を絡ませて転んでしまう。燕は必ず外で糞をするので、コシケイ以外で燕を嫌う寮生はおらず、燕がコシケイから逃げられるように寒くても廊下の窓を開けてやる。

  夜は燕が一番好きな時間だ。夜になるとどこからともなく寮生が集まって中庭の花壇で焚き火を始める。民青池の向こうで小枝を拾い焚き付けにして、薪は昔切り倒したC棟北の大銀杏を干したものを使って、ほんの五分で立派な炎が昇がる。燕はまるで自分がマッチを擦っているように、寮生の動きに合わせて自慢の尾っぽを振る。火が付くと燕は花壇の縁に座って冷えた体を温める。花壇の周りには寮生が入れ替わり立ち替わり座って、本を読んだり、話し合ったり、一人でチェスを指したりする。燕は寮生と喋る日もあれば、誰が話しかけても黙って火を見つめているだけの日もある。いずれにしても燕はバケツの水でじゅっという音とともに火がしっかりと消えるまでそこにいる。火が消えたらC棟四階踊り場に戻って共用の食器棚の片隅の土鍋の中で眠りに落ちる。

  僕と燕が初めて喋ったのは年末で多くの寮生が実家に帰り始め、だんだんと寮がひっそりとしてきた頃だった。僕は同部屋の寮生に誘われて焚き火に当たりに中庭に出た。そこでB3の寮生と話していたが、夜の九時を過ぎると「寒いから」と言って二人が帰ってしまい、一人になった。僕は、燕が澄ました顔をしているのと、喋り足りないのとで、燕に意地悪を言いたくなってしまった。\\
  「寝坊で仲間に置いて行かれるなんて、寮生みたいだね。寮生も朝寝坊だし集団行動ができない」\\
  「寮生はそれで何も失わないじゃないか。昼から起きて麻雀をしておけばまた明日が来るのさ。こっちは毎日、青森の猿が温泉に入らないと冬を越せないみたいに、こうして火に当たらなくちゃいけなくなってしまったんだぞ」\\
  燕が立ち上がってそう言うと、自慢の尾っぽをくるっとこちらに向けて糞をした。その後は何を言っても答えてくれなくなった。僕はたまらなくなって、まだ大きな薪が燃え残っているのに水で始末をつけてしまった。すると燕は飛び上がって、今度は僕の頭の上に糞を落とした。随分な燕だなと思った。

  年が明けると僕らはすぐに仲直りして、一緒に散歩するようになった。燕も僕も鴨川が好きだった。荒神橋西詰に座って、対岸の上半身裸で日光浴をするおじさんや、散歩している高級そうな犬を論評するのだ。\\
  「冬でもあの日焼け具合を保っているのはすごいね。毎日、あそこのベンチを陣取っているだけはあるね。」\\
  「石川さん家のシェパードは小林さん家のセントバーナードと仲が悪いようだ。あの二人が出会っても吠え合うような軽率なことはしないけれど、飼い主同士が睨み合うんだよ。飼い主はペットの恨みを代弁すると言うからね。二年くらい前に何かあったに違いない」\\
  「とんびという生物はとても卑しいのだ。特にあの右翼に斑点のあるやつは鴨川にカモがいないと見るや、京大キャンパスまで出張して弁当を\ruby{掻}{か}っ\ruby{攫}{さら}うんだよ」

  二月になると民青池の鳥居の上が燕の夕方の居場所になって、そこから僕の火着けに口を出すようになった。やれあっちの枝が燃えやすそうだとか、やれこの薪はまだ乾ききっていないだとか、薪の組み方がなってないだとか言ってくる。燕はとても目がいいので、僕がその通りにしないとすぐに不機嫌になって、たまに糞も落としてくる。言う通りにしてみると、確かに早く火が着く。その冬、燕はどの寮生よりも焚き火を多く見たのだ。燕の従えば雨の日だっていとも簡単に火が着いた。火が着くと燕はとても得意げな顔になって、僕の周りをくるくると飛んだ。

  そんな風にして僕らは魅かれ合い、結婚した。結婚式はいつもより盛大な焚き火の前でやった。証人はコシケイに頼んだ。寮生が四人ぱらぱらと手を叩いて祝福してくれた。燕は相変わらず給湯器下のごきぶりを食べているのだけれど、早く春になって中庭に虫が帰って来て欲しいと僕は思う。ごきぶりからできた糞を浴びせられるのはもう御免だ。



