

\section{議論改善PT座談会}
\label{sec:gironkaizen}
\bunsekisha{文責}{05のこたつ}

熊野寮は生活の場であると同時に、学生が自主的に管理運営する自治空間でもある。そして、そんな自治の根幹をなすのが、寮生間の議論だ。定期的に開かれる会議、喧々諤々の議論、熱き論戦などなど…この寮で議論が尽きることはない。「議論改善PT」は、こうした寮の議論をよりよくするために活動する有志団体である。議論改善PTの宣伝も交えつつではあるが、寮生のリアルをのぞいてみてほしい…。

\subsection{登場人物紹介}

\begin{table}[h]
    \begin{tabular}{lp{13cm}}
        \hline
        \vspace{2mm}
\textbf{05のこたつ}
&
入寮4年目。法学部4回(新社会人?)。執筆時現在、卒業まで2単位足りていないので、本当に新社会人なれるか分からない。標準語と関西弁がまじりがち。議論改善PT創設者。議論改善PTについてはたぶんこいつが一番詳しい。
\\
\vspace{2mm}
\textbf{寺さん}
&
入寮4年目。工学部工業化学科4回生(新M1)。献血と謎解きが趣味。実は2020年度から入寮パンフに存在し続けているので探してみよう。
\\
\vspace{2mm}
\textbf{安倍晋三}
&
新入寮生。工学部物理工学科4回生(工学研究科材料工学専攻、新M1)。寮内では、「異次元の金融緩和」を続ける日銀が、インフレ局面で直面する危機について論じている。
\\
\vspace{2mm}
\textbf{パワー}
&
新入寮生。教育学部1回(新2回)。転学部したいらしい。寮では頭を使うパワー系として、同期に恐れられている。
\\
\vspace{2mm}
\textbf{ただの阿呆}
&
入寮4年目。文学部4回生(卒論と院試が終われば文学研究科新M1)。日本史を勉強して、寮の史料整理もしたいと思ってる。時々料理を作ってはそれで少々のお金と笑顔を得ようとしている。
\\
\vspace{2mm}
\textbf{武蔵野アブラ学会}
&
入寮3年目。経済学部3回(新4回)。寮内に友達が全くいないが、元気に生きている。05のこたつと同部屋。\\
\hline
    \end{tabular}
\end{table}


\begin{multicols}{2}

(武蔵野遅刻中)



\talker{05のこたつ}それでは始めていこうかな。司会は僕がやりまーす。拍手承認で。

(一同拍手)\footnote{寮の会議では議長は拍手承認で決まることが多い}



\talker{05のこたつ}こちら本日のお品書きです。



・近況報告

・PTとは

・議論改善PTとは

・PTに入ったきっかけ



\subsection{近況報告}

\talker{05のこたつ}じゃあまず最初は近況報告から〜最近、談話室\footnote{各ブロックに1つある交流スペース}でこんな遊びしてるよなど、なんかある?

\talker{寺さん}では私から。スプラトゥーンのランクがS+になったから、エックスマッチに参戦できるようになりました!

\talker{パワー}すごそう。

\talker{安倍晋三}なにかわからんけど。

\talker{寺さん}S+はほぼ一番上のランク。B4\footnote{B棟4階居住区のこと}の中で最強になってしまった。

\talker{一同}お〜

\talker{寺さん}ちなみに研究室は1か月ブッチしている。

\talker{パワー}それいいんですか?

\talker{寺さん}ダメ

\talker{一同}(笑い)

\talker{05のこたつ}ただの阿呆は卒論\footnote{卒業のために必要な論文。卒論が書けなくて留年する人もたまにいる。法学部と経済学部には卒論がないぞ!ヤッタネ☆}があるんやんな。

\talker{ただの阿呆}卒論きつい…というか締切が早い。

\talker{安倍晋三}寺さんは卒論大丈夫なん?

\talker{寺さん}実験データは集め終わっているので。

\talker{一同}お〜

\talker{05のこたつ}パワーはどう?僕がパワーですごい印象に残ってるのは、やっぱKMN\footnote{熊野寮発、自治やっちゃう系アイドル。2010年結成、13年目に突入している}かな。

\talker{パワー}そうですね。NFでKMN踊って、そのあと寮祭\footnote{熊野寮祭のこと。11月末から12月初旬にかけて行われる。地球で一番楽しい祭典}やって、寮生大会\footnote{半年に1回ある寮の会議。寮の最高意思決定機関。全寮生に出席義務がある}やる、って感じでした。

\talker{ただの阿呆}寮祭は何やってたんだっけ?

\talker{パワー}パンフ責\footnote{寮祭パンフレットの責任者}やってました。

\talker{ただの阿呆}パンフ大変やったね

\talker{パワー}大変でしたね

\talker{寺さん}パワーはこれ(議論改善PT)にも参加してるし、すごいよね

\talker{05のこたつ}今、何個いってるの?会議とか。ご飯のおともPTもやってるよね

\talker{パワー}そうすね。ご飯のおともPTやってて、寮外連携局\footnote{寮外の学生と連携するイベントなどを企画する組織}やってて、あとーこれ(議論改善PT)と、あと広報局\footnote{寮外向けへの広報を行う組織。熊野寮Vtuber「熊野あじり」の運営を行っている}、それで部会が文化部\footnote{寮内のコンパやイベントなどの催し事を管轄する部会}と、あと人権擁護部\footnote{ハラスメントの対応や消防の取組み、喫煙所管理などを行う部会}もなんかやってて

\talker{ただの阿呆}え?

\talker{05のこたつ}数がおかしいw

\talker{パワー}あと委員会が入選\footnote{入退寮選考委員会の略。入寮パンフの作成は入選がしきっている}、でまぁ資料委員会\footnote{寮のコピー機の管理や、会議資料の校閲、検討会の運営を行っている}も流れで入ってる

\talker{パワー}あとMUC\footnote{音楽室利用者会議のこと。年に何回か音楽ライブを開催している。}も

\talker{安倍晋三}つぎ、わたし?んー…近況報告ムズイな.

\talker{寺さん}寮生大会の話は?

\subsection{寮生大会の話}

\talker{05のこたつ}寮生大会の話いいね。僕はこの前の寮生大会で副議長やりました。

\talker{ただの阿呆}05とパワーが副議長で、寺さんが議長か。今回は荒れましたね。

\talker{05のこたつ}荒れない寮生大会はない。大会では、安倍晋三くん大活躍やったやんな。

\talker{寺さん}いっぱいアピール\footnote{寮生大会で全寮生に向けて、寮自治に関する取組みの宣伝を行うこと}してたね

\talker{安倍晋三}アピールしました!というか、今回の議長団の数、多くなかった\footnote{直近の寮生大会の議長団は、議長が2人(前半と後半に1人ずつ)と副議長などが6人で、計8人体制だった。}?

\talker{ただの阿呆}そうね。まぁ最近の寮生大会は、もう、特定の議長と副議長が全ての議題を扱うことってないよね?

\talker{05のこたつ}実はそれ、議論改善PTの活動によるものなんですよぉ〜!(ドヤ)

\talker{一同}おぉ〜!(拍手)

\talker{寺さん}な、なんだってぇー

\talker{05のこたつ}まず、昔がどうだったかを話すと、昔は議長団が2,3人とかで、マンパワーが不足してたんだよね。

\talker{寺さん}議長団が途中で疲弊すると終了する、というのは何となく皆さんお分かり頂けるのでは。議長団が議論をまとめられなくなると、混沌が発生して…おわり。

\talker{安倍晋三}そして収束の見えない議論へ、になっちゃう。

\talker{05のこたつ}なので、議論改善PTが、議長を交代制にしたりとか、副議長を多人数設ける、とかを2〜3年前からやったんですよぉ〜。

\talker{安倍晋三}つまり議論改善PTの成果ってこと??

\talker{05のこたつ}そう!議論改善PTの成果なんですよねぇ〜!(ドヤ)

\talker{寺さん}す、すごい!

(茶番終了)

(武蔵野アブラ学会到着)

\subsection{PTとは}

\talker{05のこたつ}既に「議論改善PT」って我々言ってると思うんだけど、このパンフ読んでる人って、PTってなぁにっていうのが分かってないよね。

\talker{安倍晋三}説明むずそ〜

\talker{ただの阿呆}Projectteam☆

\talker{05のこたつ}そう。PTっていうのは、Projectteamの略で、寮で何かしら問題意識をもつ人がいた時に、その問題意識を共有する有志の寮生を集めて、寮の問題解決に動く組織、みたいなものです。

\talker{安倍晋三}すごいまとまってる。考えてきましたみたいな笑

\talker{05のこたつ}まぁね笑笑

\talker{寺さん}じゃあPTって他にもあるんですか??(茶番)

\talker{一同}(笑い)

\talker{05のこたつ}笑笑いい質問ですね

\talker{ただの阿呆}す、すごい。まるで台本があるかのようだ!(補足:ないです)

\talker{安倍晋三}ごはんのおともPTは?

\talker{パワー}あれはPrettyteamなので違います

\talker{一同}ややこしいな〜ww

\talker{05のこたつ}他にいま活動してるのは?

\talker{寺さん}間違いなく活動しているのは、おそうじPT\footnote{寮内の掃除をするPT}

\talker{安倍晋三}中銀破綻PT\footnote{インフレ局面下の日銀が異次元緩和の出口において取り得る金融政策の自由度について調査するPT。寮の食料備蓄を行う備蓄局の中に設置されている。}

\talker{パワー}冷房PTって一瞬ありませんでしたっけ?

\talker{一同}あ〜

\talker{05のこたつ}あー名前なんだっけ?

\talker{安倍晋三}食堂エアコンPT\footnote{真夏の食堂がくそ暑いので、食堂にエアコンを設置しようとしていた}

\talker{パワー}あ〜食堂エアコンPTか

\talker{05のこたつ}あと、お風呂PTでしょ。寮生にお風呂入らせるやつ笑

\talker{ただの}あれPTなの?

\talker{安倍晋三}でも、お風呂PT活動実態あるからな〜。風呂に入ってない寮生に…

\talker{パワー}恫喝してシャワーカード\footnote{熊野寮のシャワーを利用するにはシャワーカードが必要。3分10円で利用できる。}を…

\talker{安倍晋三}そう、シャワーカードを貸すっていう

\talker{05のこたつ}シャワーカード貸してんの!?w

\talker{安倍晋三}シャワーカードとシャンプーを貸す活動を行ってる

\talker{ただの阿呆}それでシャワーカードを失くされる

\talker{一同}(笑い)

\talker{安倍晋三}かわいそうに

\talker{05のこたつ}すごい、めちゃくちゃ慈善活動してるやんお風呂PT

\talker{寺さん}「お風呂入らない人コンパ」\footnote{お風呂に入らない人でこたつに入り、コンパをするという寮祭企画}を粉砕しようとしてた

\talker{05のこたつ}寮祭であったな、その企画笑

\talker{ただの阿呆}あのコンパ、あまりにかわいそすぎて、銭湯の回数券一枚あげた

\talker{武蔵野アブラ学会}俺もカンパしましたよ、1000円くらい

\talker{一同}(笑い)

\talker{05のこたつ}1000円!?w

\talker{安倍晋三}めっちゃカンパするやんw

\talker{武蔵野アブラ学会}北風と太陽ですよね

\talker{安倍晋三}なるほど、太陽側ね笑

\talker{パワー}(カンパ)3000円ちょっと集まったみたいですね

\talker{一同}すご!

\subsection{生活習慣の話}

\talker{安倍晋三}最近、私も2日に1回くらいしか風呂入れてない気もする

\talker{パワー}僕も2日に1回くらいしか外に出なくなりましたね

\talker{一同}(笑い)

\talker{安倍晋三}おかしいな笑 平日って7日中5日あるんだけど

\talker{ただの阿呆}ご飯はどうしてんの?

\talker{パワー}寮食あるんで

\talker{安倍晋三}寮食あるから出なくてよくね?以上。

\talker{05のこたつ}あ、僕も外出てないわ。全然授業ないし

\talker{ただの阿呆}そうだ。お前、後輩と見に行く映画さぼっただろ

\talker{05のこたつ}起きれへんくてさ〜、今日午後3時起きで寝ブッチ\footnote{寝坊により、授業などに行けなくなること}しちゃった、、

\talker{安倍晋三}やばwひどい先輩や

\talker{武蔵野アブラ学会}この人今日8時すぎとかに寝てましたからね

\talker{05のこたつ}眠れへんかってんてぇ〜

\talker{武蔵野アブラ学会}俺がなんか8時くらいに寝てる時に寝てなかったですからね

\talker{05のこたつ}そうそう、寝てなかった。てか朝寮食\footnote{パンと牛乳、ゆで卵、紅茶が170円で食べられる。安価!おいしい!それが寮食}食ったもん、俺今日

\talker{寺さん}たぶん僕と完全に生活リズムが一緒だと思う。朝9時睡眠、夕方18時起床みたいな

\talker{05のこたつ}あ〜めっちゃ近い。それと1時間差くらいやわ

\talker{武蔵野アブラ学会}2人で見に行く予定をとばしたんですよね?

\talker{05のこたつ}そう、2人で見に行くのとばした、、

\talker{一同}(笑い)

\talker{安倍晋三}やばすぎw

\talker{ただの阿呆}ほんとにかわいそうw

\talker{05のこたつ}寝ようと頑張ったけど眠れなかったのぉ〜

\talker{武蔵野アブラ学会}俺今日1人で見に行く映画、8時に寝たけど頑張って12時に起きて見に行ったんですよ

\talker{寺さん}議論改善PTならぬ生活崩壊PTじゃん

\talker{安倍晋三}私も最近午前中に研究室に行けなくなってきている

\talker{一同}(笑い)

\talker{寺さん}議論は改善できるけど生活は改善できない笑

\talker{武蔵野アブラ学会}どっちかですからね、基本的にね

\talker{安倍晋三}トレードオフの関係にあるん?w

\talker{05のこたつ}そんなことある?w

\talker{ただの阿呆}議論を改善すべきか、生活習慣を正すべきか、それが問題だ。

\talker{05のこたつ}生活習慣でいうと、昔は寮生大会って夜21時開催で…

\talker{寺さん}そう!笑

\talker{05のこたつ}いま急に議論改善PTの話に戻ってきてるよ〜

\talker{寺さん}生活習慣といえば〜!??

\talker{一同}(笑い)

\talker{05のこたつ}笑笑そう、生活習慣といえば、もともと寮生大会は金曜夜21時開催で翌朝4時、5時までかけて徹夜で議論してたんですよぉ〜

\talker{ただの阿呆}うわぁ

\talker{安倍晋三}生活習慣といえば寮生大会っていうww

\talker{パワー}今は何時なんですか?

\talker{05のこたつ}今は土曜の14時か16時

\talker{パワー}へぇ〜(茶番)

\talker{一同}(笑い)

\talker{05のこたつ}今の寮生大会の時間は知ってるでしょ笑

\talker{ただの阿呆}どうしてそうなったんですか??(茶番)

\talker{05のこたつ}誘導尋問がすごい笑 丁寧な誘導や笑 そう、もともと寮生大会で徹夜で議論してたことに問題意識をもった僕とかその時の代の議論改善PTの人たちが、徹夜の寮生大会おかしいよね、っていう発信を行ったり、事前検討会\footnote{議題について関心のある寮生を集めて議論を行う会議体}を推進して寮生大会の議論が過度にもめないようにしたんですよぉ〜!昼開催の提起をしたのは、議論改善PTの議論を引き継いだ選挙管理委員会\footnote{選挙の運営を担う委員会}からやけどね。

\subsection{議論改善PTとは}

\talker{05のこたつ}ここまで色々話したけど、次がようやく「議論改善PTとは」なんよね笑。議論改善PTとは、寮議論に問題意識をもつ寮生が集まった有志団体で〜す。

\talker{安倍晋三}議論改善PTの最初の出発点が、寮生大会の議論をやりやすくしましょうみたいなところから始まったっていうのは、知らんかった

\talker{05のこたつ}昔は、今よりも寮生大会が夜遅くにやってたってこともあって、議長も意見者もみんな頭が回ってない…感じやった。

\talker{安倍晋三}ルーズルーズってこと?

\talker{寺さん}ルーズルーズ…?

\talker{05のこたつ}一部の生活習慣がずれてる人だけが発言してて、それ以外の人は食堂で寝たり、部屋に帰ったりしてた。みんな疲れてる中で議論してるから、同じ議論を繰り返したり…

\talker{寺さん}終わりやった

\talker{パワー}一回見てみたい

\talker{寺さん}議事録が寮内に残っているので見ると分かるんですけど、キャッチボールしてると思ったら、お互いボールたくさん抱えてお互いに対して玉入れしてる感じになってる

\talker{ただの阿呆}収拾がついてない

\talker{05のこたつ}入寮パンフをご覧の皆さんは、2019年の入寮パンフの『寮議論で現れがちなKUMANeverまとめ』を是非ご覧ください!

\talker{寺さん}ブヤコフ=マクシモヴィッチ!

\talker{安倍晋三}2019年の入寮パンフってアクセスできなくない?

\talker{05のこたつ}実は、熊野寮のHPから見れるんですよぉ〜

%\url{https://kumanoryo-pamphlet-2019.netlify.app/chapters/%E5%AF%AE%E7%94%9F%E3%81%AE%E5%A3%B0/%E5%AF%AE%E8%AD%B0%E8%AB%96%E3%81%A7%E7%8F%BE%E3%82%8C%E3%81%8C%E3%81%A1%E3%81%AA%E8%AB%96%E8%80%85kumanaver%E3%81%BE%E3%81%A8%E3%82%81}
%urlが長いのでどうにかしたい

\talker{寺さん}え〜そんなぁ〜!(茶番)

\talker{一同}(笑い)

\talker{05のこたつ}話を戻すと、昔の寮生大会は今の比ではないくらいに紛糾してたんよね

\talker{寺さん}5時6時の畳のエリア、やばいよ

\talker{ただの阿呆}みんな寝てる

\talker{05のこたつ}みんなもう死屍累々みたいな感じ。まぁエンタメとしては面白かったけど、寮の最高意思決定機関がそういう状態ってのはよくないなと思うよね

\talker{寺さん}最後まで立ってたやつが勝ち、やから

\talker{05のこたつ}体力勝負、パワー勝負になってしまうからね

\talker{寺さん}あれのためだけにぼく一回、生活習慣破壊してた笑。破壊してから臨んだ方が議論できるなぁと思って。水曜あたりから破壊して〜

\talker{安倍晋三}寮生におかしなインセンティヴ働いてるw

\subsection{議論改善PTに入ったきっかけ}

\talker{安倍晋三}あ、でも、議論改善PTって一回活動が下火になったんじゃなかったっけ??(茶番)

\talker{05のこたつ}いい質問ですね〜笑

\talker{一同}(笑い)

\talker{安倍晋三}だんたん寮議論がよくなってきているという単線的な史観は間違っているのではないかと

\talker{05のこたつ}PTはしょせん有志団体なので、モチベを持ってる人がいるかいないかで活動の規模感が左右される。議論改善PTが発足した2019年は先輩も参加してくれてて5,6人くらいいたんやけど、2020年以降は僕1人で活動してたんよね。2022年になってから、寮議論に関心のある安倍晋三が入って来てくれて(心強すぎ)、僕もこのPTを残そうという気持ちになってさらに人を誘ったりした。安倍晋三くんは議論の仕方みたいなところに興味をもって入ってきてくれたんやんね?

\talker{安倍晋三}私は秋の入寮オリテ\footnote{入寮に際して受講が必要なオリエンテーション。寮生になる上で必要な情報をインプットする}で議論の仕方みたいなスライドを作って新入寮生向けに発表したんですけど、なんかまぁ熊野寮に入って、2陣営あってお互い対立してないのになぜか対立しているかのようになっていて不毛な議論をずっとしているという光景があるなぁと思ってて、こうすれば両者とも満足できますよというのを誰も提示しないから紛糾しているという問題意識があったね

\talker{05のこたつ}みなさんは入ったきっかけありますか?ぼく、パワーってどういう経緯で入ったか覚えてないんよね。いつの間にかしれっと入ってた印象

\talker{パワー}ぼくは、某検討会で副議長やったので、そこからですね。最初は、SCで「若い人ー?」って言われて。SC「若い人ー」って言いがちなんですけど、笑副議長として検討会の議論の内容を検討会に来ていない人にも周知する必要があって議論のまとめを作ってました。

\talker{05のこたつ}ありがてぇ〜。武蔵野アブラ学会も副議長やってくれたやんな

\talker{武蔵野アブラ学会}あーあのとんでもなく締切やらかしたやつね!

\talker{05のこたつ}あ、そうやったっけ?

\talker{パワー}第3回まとめと第4回まとめが同時に出てました

\talker{05のこたつ}武蔵野アブラ学会は就活のためにPTに参加してくれてるもんね

\talker{一同}(笑い)

\talker{武蔵野アブラ学会}学チカ\footnote{学生時代に力をいれたこと、の略。就職活動の面接で聞かれることが多い質問の1つ。}ないわーって1日10回くらい言ってたら、あるよ!って言われて来たのがこれって感じですね笑

\talker{05のこたつ}コロナで学チカなくなった世代やもんな

\talker{一同}あ〜

\talker{寺さん}コロナで学チカしか湧いてこなかった

\talker{05のこたつ}コロナ対応系ってだいたい学チカになりそう

\talker{寺さん}コロナ対策PTやってたから

\talker{05のこたつ}熊野寮ってあんま意識してないけど、学チカの宝庫やんな

\talker{寺さん}いま選び放題

\talker{武蔵野アブラ学会}1個くださ〜い

\talker{一同}(笑い)

\talker{ただの阿呆}寺さん1つあげたら?笑

\talker{パワー}まぁ、嘘言ってもばれないですからねぇ

\talker{武蔵野アブラ学会}そう、まじでね、嘘言ってもばれないからなぁ

\talker{ただの阿呆}でも入社してから、能力のなさが露呈してしまう

\talker{安倍晋三}まぁ誰も覚えてないからな、そんなこと

\talker{一同}(笑い)

\talker{寺さん}論点がずれてる笑議論改善PTが議論破壊されてる

\talker{05のこたつ}ほんまや

\talker{武蔵野アブラ学会}ま、副議長はいい経験でしたね

\talker{05のこたつ}寺さんとただの阿呆は僕が誘ったんやったっけ?

\talker{寺さん}僕は純粋に活動内容には共感してたんやけど、2回生3回生の時が寮の仕事で忙しすぎて入ってはなかった感じかな。レポートがあって死んでた時期もあった

\talker{ただの阿呆}工化…

\talker{05のこたつ}そうなんや〜

\talker{安倍晋三}ただの阿呆は?

\talker{ただの阿呆}私は、安倍晋三に誘われた。自治会憲章の改正できるよって呼ばれたら、気づいたら変なところにとばされてた笑

\talker{一同}(笑い)

\talker{05のこたつ}エクストリーム帰寮\footnote{車で遠方まで運ばれ、そこから歩いて帰ってくることを求められる寮祭企画。}みたいなww

\talker{寺さん}30km離れた山奥に連れていかれてた、みたいな

\talker{安倍晋三}じゃあ歩いて帰ってきてください

\talker{ただの阿呆}ひどい。もう2回やってるからいいよ、あれは

\subsection{まとめ}

以上、議論改善PT座談会でした。誕生してから4年目となった今年も、寮議論についてメンバー各々の問題意識を形にしていく場であり続けられたらいいなと思います。新入寮生の皆さんも、寮で問題意識が湧いたり、やりたいことができたら、積極的にPTを作って活動してみましょう!

\end{multicols}


