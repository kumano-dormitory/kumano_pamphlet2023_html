\subsecnomaru
%\subsectionを◯なしに

\section{すきあらばけいおん}
\label{sec:suki_kon}

\bunsekisha{文責}{京大全学けいおん斗争委員会(京大全K斗)\quad twitter\url{@sukiarabakeion}}

\subsection{0.ごあいさつ}

\emphbf{
寮祭企画「すきあらばけいおん」\\
熊野寮祭。食堂モニターが空いているすきを見つけて「けいおん!」を見よう。唯の生誕祭をしよう!豊郷行こう!左京区を歩こう!
}

入試お疲れ様です!去年の熊野寮祭では、こんな企画があったとさ。この文章ではけいおんの名シーンを解説するよ!Twitterでもけいおんの名シーン解説をやっています。京大生と桜高生の生活圏はほぼ一致してるので、けいおんを見ると毎日の通学が聖地巡礼になって楽しみになりますよ。

\subsection{1.団結の力}

\begin{quote}
  \emphbf{
    革命的りっちゃん「どんな理由があろうとな!歌ったらアカン歌なんかある訳ないねん!うちらの団結があれば相手が国家権力でもおんなじや!うちら、友達もせんせいもみんな皆に団結拡大して、機動隊なんかデコピンで倒せんねん!」

    反革命むぎちゃん「まあまあ、落ち着いて…お菓子はいかがですか?」

    (OVA「けいおん!免許合宿・京北鹿ロードキル編」より)
  }
\end{quote}

ライブでイムジン河を歌おうとしたHTTを弾圧する桜高当局。りっちゃんが弾圧に対してキレたのを止めようとするむぎちゃんはここではナンセンスであると思います。りっちゃんの決起は正当な怒りに根拠づけられたものです。お菓子で懐柔するのは反革命ですね。でも、こんな状況では誰でも咄嗟にむぎちゃんみたいな行動を取っちゃいそうですよね。これをちゃんと総括してHTTはもっと大きな団結を生み出すバンドになるんです。むぎちゃんは階級的視点を手に入れて、実践します。悪いブルジョワになんか絶対にならないんです。それくらい団結の力は強いよ!

\subsection{2.オルグ}

\begin{quote}
  \emphbf{
    唯へ

    久しぶり。元気にしてる?

    12月9日に京大で、大学当局による学生への懲戒処分に反対する集会をします。唯も職場の組織化で忙しいと思うけど来ない?全国から仲間が集まって発言してくれるから、きっと唯にとっても学ぶことが沢山あるはず。ぜひ来てください。

    真鍋和

    \tatespace
    曽我部先輩 

    ご無沙汰しております。先輩はいかがお過ごしですか?私が熊野寮に入り、活動家になって早いもので13年が経ちました。入寮した翌年の学生ゼネストに突撃してくる機動隊を見て私は衝撃を受けました。「普通に」生活していたときには見えざるものだった権力が、階級闘争の前線に立った瞬間に剥出しの暴力でもって迫ってくる、破壊的な経験。私はそれ以前も社会に関心を持っていたつもりでしたが、権力が「見えざるもの」である限り、私も秩序の中で踊らされていたことに気づいたのです。でも、そこから闘いに決起するまでに長く苦しい時期がありました。…

    (中略)

    またお会いしましょう。 

    真鍋和

    (OVA「けいおん!全学自治会再建編1」より)
  }
\end{quote}

12月26日はゆいちゃんの幼馴染であるのどかちゃんの誕生日でした。桜ヶ丘高等学校卒業後、K大法学部自治会の委員長を務め、熊野寮に入寮して、バリケードストライキで放学処分を受けてなお斗いつづける我らが真鍋和同志の書いた手紙です。


\subsection{3.非和解性}
\begin{quote}
  \emphbf{
    平沢唯「12月集会で職員に『金持ち学生ばっかなんやろ、羨ましいわ』と面と向かって煽られて、久々に切れた。こっちは学費払えんくてどんだけ不安やった思とんねん。まじで次見たら張倒してまいそうやわ。中間管理職か何や知らんけど、へらへら話してきよる時もこっちは処分されるかもしらんねんぞ。」
    (OVA「けいおん!全学自治会再建編2」より)
  }
\end{quote}

唯ちゃんが集会で職員の挑発を受けるシーンですね。職員と馴れ合うことはあまり得にならないのでやめた方が良いということを唯ちゃんはこの時はっきりを悟ったようです。「非和解」という言葉は、階級間の対立の投射にのみ使いましょう。日常で不快なものに、いちいち非和解と言っていたらあっという間に分断されちゃいます。それこそ権力の思う壺ですね。


\subsection{4.確信}
\begin{quote}
  \emphbf{
    平沢唯「以前は私はニヒリストだと思っていた。でも、内省すると、ばらばらの階層の信念が私にも存在することを観察した。しかし、いざ、それに基づいて実践しようとするとき、私はそれらの信念を一体系に配置するところのものを知らぬことに気付き、おののく。どうすればよいか?」

    琴吹紬「体系というものは生きていくのに必要なものだ。しかし、それ以上に役割があるのか疑うことができる。全体を生きられない人間にとって似つかわしいものが体系。選択の問題だ。」

    秋山澪「それでは相対主義に出戻りするだけではないか!私の信念が信念たるには、他者との共有可能性、もしくは少なくとも行動において大きな障害を産まない程度の伝達可能性、もしくはそれを少なくとも日常のレヴェルで前提していて困らない程度のそれがないといけないんだ…。」

    (映画「けいおん!根源悪編」より)
  }
\end{quote}

自分の思想を検証しようとして相対主義に陥った唯ちゃんが回復してくる治療の過程を描く新作映画のワンシーンですね。こういうのって2回生くらいで悩む人が多いイメージがありますね。

\subsection{5.革命的展望}

けいおんのみんなが京大にも息づいていることを感じたことでしょう。今年の寮祭でももちろん「すきあらばけいおん」はおこないます。みんなでけいおん見ようね。

\subsecdefault
%\subsectionのデザインを元に