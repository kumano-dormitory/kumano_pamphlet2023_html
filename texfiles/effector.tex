
\subsecnomaru
%\subsectionをタイトル表示を◯なしに

\section{僕のエフェクターボード大紹介!!!}
\label{sec:effector}

\bunsekisha{文責}{テキーラ}

どうも、テキーラです。ギターを弾いているとエフェクターボードを組みたくなりますよね。僕もエフェクターボードを組んでいます。てことで僕のボードを紹介したいと思います。


\subsection{①VOX V845:ワウペダル}

ワウペダルの中でも安くスタンダードなモデル。ワウで有名なのはVOXとJIM DUNLOPのCRYBABYだと思うんですけど、CRYBABYは高音のかかり方がエグすぎて耳が痛くなってしまうのでこちらにしました。本当はXoticのXW-1(King Gnuの常田が使ってるやつ、ワウのかかり方を調節できる)が欲しかったんですけど、3万円は高すぎて断念。ただ、このワウは高域のかかり方が逆に物足りないのでお金がたまったらXW-1を買いたいと思ってます。

\subsection{②XOTIC SP Compressor:コンプレッサー}

本当はコンプレッサーは買う予定ではなかったが、原価2万円のところハードオフで7,000円で売っているのを見て即購入。ハードオフは頭おかしい値段で中古品を売っているのでおすすめです。僕は普段あまり強くはコンプをかけないで、自然にサステインを伸ばすように使っています。あとこのエフェクターのいいところは通すだけで音が太くなることで、これがないと音がすかすかに聞こえてしまいます。また内部にスイッチがあって細かい調節も可能。最初に置くだけで音も良くなるしコンプもかかるので超おすすめです。

\subsection{③Effects Bakery Bagel OverDrive:オーバードライブ}

格安エフェクターブランドEffects Bakeryのオーバードライブ。なんとお値段4,500円弱。それで割と本格的な歪みが手に入るのだから買って損はないです。僕はTS系と聞いていたので後段の歪みエフェクターのゲインアップに使おうと思って買いました。でも少し物足りないかな~。

\subsection{④VOX VALVENERGY SILK DRIVE:オーバードライブ}

一番最近手に入れたペダル。新開発された真空管のNutubeが搭載されていて、チューブアンプのようにVolumeへの追従性が良い歪みを手に入れることができます。オーバードライブとしての分類はダンブル系らしいです。確かに冷たい歪みっていう感じではありますね。逆に僕がその歪みを使いこなせていないです。僕はこのペダルを踏みっぱにしてクランチを作ってます。あとこのエフェクターで特徴的なのがエフェクター真ん中の画面に音の波形が表示されることです。今作ってる歪みがどの程度歪んでいるのかを視覚的にとらえることができます。正直歪み量なんて見てもわかんないんですけど、かっこいいのでジャケ買いもありです。

\subsection{⑤BOSS BD-2 Blues Driver:オーバードライブ}

超有名なBD-2です。安いのにプロでも使ってる人結構いますよね。ナンバガの田渕ひさ子とか。歪みにバイト感があって気持ちいいです。またこれもSILK DRIVEと同じようにVolumeへの追従性が良く、弱く弾けばクリーンサウンドを出すことができます。音としてはジャキジャキとしたシングルコイルに合う歪み。オルタナ向きですね。僕はかなり歪ませてディストーションとして使っていますが、クランチ作ったりとかブースターとして使うのがこのエフェクターはいいと思います。ちゃんとしたディストーションが欲しい。

\subsection{⑥Effects Bakery Cream Pan Booster:ブースター}

またもや格安ブランドのEffects Bakeryから。歪みの後に繋いで音量をあげるクリーンブースターとして買いました。VolノブのほかにBrightスイッチがついており、音のキャラクターを決めること⑦ができます。僕はBrightスイッチはオンで音抜けをよくさせていますが、少し耳が痛くなってしまうかなとは思います。EP Boosterが早くほしいっす。

\subsection{⑦ZOOM MS-50G MultiStomp:マルチエフェクター}

僕が最初に買ったエフェクター。初心者が最初に買うのにめちゃくちゃおすすめ。見た目はコンパクトエフェクターなのに172種類のエフェクトを最大6個まで同時に使うことができて、。これでどのエフェクターがどんな効果を持っているかを学ぶことができます。一年以上これだけでライブできたのでそれくらい万能です。今はモジュレーション系と空間系のために使ってます。ただこれも少し高いものに買い換えたいかな。今欲しいのはStrymonというブランドの空間系。明らかに音が良いのだけれど高いんですねこれが。5万って。金のない学生には辛いですね。

\subsection{⑧Nano Big Muff PI:ファズ}

ジミヘンが好きでジミヘンといえばファズ!ということで購入。ただ、ジミヘンが使っていたようなVolumeを絞ったらクランチになるようなものではなく、全てをBig Muffの音にする凶悪エフェクターだったので一番後ろに繋いで全てを破壊する音を出すために使っています。今やってるバンドがシューゲイザーバンドになりそうなのでちょうどよかったです。音の壁最高!ちなみにこれはBig Muffの小さい奴なんですけど、本物のBig Muff見たことあります?馬鹿みたいにでかいっすよ笑。これいれるだけでエフェクターボードパンパンになるんでこだわりが無かったらNanoのほうをおすすめします笑。

\subsection{⑨CAJ DC・DC Station:パワーサプライ}

エフェクターではないですが、一応ボードに入っているので紹介。フルアイソレートなのでデジタルとアナログのエフェクターを同時に繋いでもノイズが乗らなくていいです。僕はSound Design Lab.というYouTubeチャンネルでおすすめされていたので買いました。このチャンネルは結構ズバズバと意見を言うので信用できますし面白いです。


\subsection*{}
以上です!一応繋いでいる順番に紹介したのですが、こっちの順番のがいいだろなどの意見がある人はぜひ熊野寮に入って僕に教えてください。むしろ教えてほしいっす。では、つたない文章でしたが読んでくださりありがとうございました!

\subsecdefault
%\subsectionをもとに戻す
