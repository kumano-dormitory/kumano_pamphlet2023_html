\section{たばこの話}\label{sec:tabako}
\bunsekisha{文責}{銅鑼みがき}

\epigraph{\sshatai{20歳未満の者の喫煙は、法律で禁じられています。}

\sshatai{喫煙は、動脈硬化や血栓形成傾向を強め、}

\sshatai{あなたが心筋梗塞など虚血性心疾患や脳卒中になる危険性を高めます。}}
{たばこ事業法施行規則第36条注意表示より}


たばこを吸うようになってしまった。喫煙というのはとても左京区的な行為で、つまり極彩色の怠惰に目を眩ませるような記号的行為だと断じていたので手を染めるつもりはなかったのに。左右の肺にたいしても、申し訳ないとまでは思わないが気の毒だとは思う。毒の気体だものね。なお、断っておきたいが、以下の戯言は喫煙を勧めるものではなく、私が禁煙を意欲してすらいないことのただの言い訳である。

入寮パンフレットという媒体を鑑みて、寮の喫煙事情をわずかながら述べておこう。寮の唯一の喫煙所は玄関前にあって、灰皿やライターやその他諸々が転がっている。灰皿はときどき入れ替わり、たまに灰皿ではないのだろうなという器もある。最近は、たぶんお香を焚く用の器が使われていて吸い殻を突っ込む穴が狭い。喫煙所には、学年や住んでいるブロックなんかの属性や所属を跨いでなんとなく緩いコミュニティが形成されていている。私がその界隈に溶け込めているかは微妙で、というのは私がたばこを吸うのは色々と切羽詰まった明け方だとかの余裕がないときが多く、そういう時間は大抵喫煙所に人はいないし、いたとしても私は人と喋る能力を失っているからだ。しかし、そんな人間でも居られるのでありがたい。

大学内では、文学部東館の中庭が良い喫煙所だ。腰を掛けられる石材があり、顔を上げると四角い空が見える。最近(一月現在)では、向日葵みたいな花が季節外れに明るい青いペンキの塗られた壁の前に咲いていた。あのニセ向日葵(ほんとうに向日葵かもしれない)が幸せであれと申し訳程度の祈りを吐き出して、中庭を出る。目の前を自転車が横切る。喫煙所の外は人々がせわしなく移動を繰り広げているので注意を取り戻さねばならない。

もともとたばこのにおいは好きだったが、自分で買って吸うほどではなかった。こんなに高いものを日々必要としてしまっては大変だ。今も毎日は吸わない。たまに吸いたくなって吸う。たばこを覚えてしまったのは他でもない、卒論のせいである。頭が言葉でいっぱいになってしまったときに、外の空気と共に煙を吸い込むと軽くなる。俳句好きの寮生から借りた句集に「文芸を吸ひこみし日や草いきれ」という句があった。かっこいいねぇ。書きかけの論と積みあがった文献の山を前にして、世界の全てが言葉と私だけのように感じられる、そういうときに肌で鼻腔で舌で感じる気体は精彩に溢れていて、私はやっと地上との関係を思い出すことができる。私にとって、その手助けをしてくれるのがたばこなのだ。卒論提出間際の時期の明け方、再生する陽を浴びながら戦友と吸い込んだ煙は今も私の胸を満たしている。

とかなんとか述べてみたが、そんな思い出なんぞなくとも、たばこは美味しい。でもそれ自体が美味しいというよりは、薬味、スパイスのような風味を足すものとして好きだ。蕎麦に山葵をそえてすするように、外気にちょっと乗せるように吸うのが美味いと思う。冬の朝の、夏の夜の、空気と吸う。雨の降り始め、本降り、雨上がりの湿気と吸う。雨の日は背徳感はひとしおだ。濡れた土草のにおいというのはそれだけでしっとりしていて素晴らしいのに、その湿っぽさに乾燥した煙をぶつけてしまうなんて勿体ない。こういうときは急がずにマッチを擦る。箱から取り出す間、ゆっくりと土のにおいをかき集め、チャッと火を灯してたばこの先に近づける。フィルター越しに火薬のツンとした香りが届く。ふぅと煙を吐きながら、炎が揺れるのをちょっと眺めてから消す。

私が選ぶのは大抵ラッキーストライクで、時々わかばだ。そういえば、父はセブンスター、父の父はハイライトだった。ラッキーストライクはとにかく美味い。有無を言わせぬ暴力的な煙感が気持ちいい。甘みの混じる香ばしさがしっかりと真ん中にありつつ、少し酸っぱくて「たばこって植物なんだなぁ」と感じられるのも嬉しい。「ラッキーストライク」という名前の能天気さ、単純な色使いのパッケージの古めかしさも好ましい。箱にこだわりはないけれど、雑な習性が災いしてソフトだとたばこが芋虫みたいにへのへのになってしまったり、鞄やポッケの中が葉っぱだらけになってしまったりするので最近はボックスにしている。わかばは、恥ずかしい話だが、好きな民族音楽家が愛飲しているので真似したことがきっかけだ。その人は不味いと言いながら吸っていたけれど、私は結構美味しいと思っている。いかにも草が燃えている感じが心地よい。パッケージも気取ってなくて可愛らしい。わけても、「わかば」の「は」の字の、丸を描いてから右下方向へきゅっと抑えられる線が魅力的だ。ただし、わかばを持ち歩くのは少し注意したほうがいい。寮生と参加していたある地域のお祭りに、とてつもなく美味しいチャイの屋台があった。お代わりを繰り返す私にチャイのおっちゃんは「たばこ一本でタダにしたるよ」と持ちかけてくれた。ありがとうございます、と私はそのとき持っていたわかばを差し出した。しかし、おっちゃんは「わかばなんて頭チカチカするん吸えるか」と言って、この取引は決裂してしまったのだ。「これだから熊野のんは」となじられた。わかばは好き嫌いが分かれるし、他の多くの銘柄より安いので、喫煙仲間と一本交換するときも少し申し訳ない気持ちになる。私はわかばを持って出るときは、二、三本のラキストも忍ばせている。ちなみに、チャイ屋では一緒にいた友達がキャスターをくれて事無きを得た。ニセモノが跋扈する左京区で、おっちゃんは本物の空気をまとっていた。今度はお店に飲みに行く。

たばこは美味しいけれど、たぶん本当に健康に悪いのだろうし、吸わない方がよいのだろう。軽い一酸化炭素中毒に陥ったときは、かなり気持ちよくてこれは確かに体に悪いと得心した。しかし、たばこでないと対抗できない毒がある。スマホである。情報中毒と言ってもよいかもしれない。私がスマホを手にしたのは高校卒業後だったが、それ以来だんだんと私は何もしないこと、ぼーっとすることがとても下手になった。「嘘でもいいから刺激がなくては死んでしま」うような飢餓感で、どうでもよいニュースをスクロールしてしまう。これでは駄目だとスマホを放り投げても、落ち着かない。身の振り方が分からないのだ。なんとも所在がない。ぼーっとすることは結構難しくて、技術が要る。特に難しいのが、視線の躾だ。どこかを見るでもなく、見ないでもなくいること。人は他人が何を見ているのかに敏感だ。何かをじっと見ている人間や、逆に何を見ているのか分からない人間に、人は不安を覚えるらしい。ふとしたときに、「どこ見てるの?」などと聞かれたことがあるだろう。何かを見ている(ように見える)ことは重大な意味を持ってしまう。電車の中で、待ち合わせの駅前で、必要以上に手元の画面を見てしまうのは、他に何を見たらいいか分からないからなのではないかと思う。きょろきょろせずに、でも放心状態には見えないように佇むことは難しい。我々は空白の時間を著しく失っている。

そこへきて、たばこは刺激中毒者が何もしない時間を過ごすためのリハビリにおいて良い補助道具となる。たばこは我々に、たばこを持つこと、たばこを吸うこと、灰を落とすことなんかを与えてくれる。何より、ゆらゆら立ち上る煙は目のやり場を提供する。息を吸い、吐く、という体の営為と、一瞬たりとも留まることのない煙のゆらめきとに意識を傾け、たまに何かを思い浮かべつつしていると、不足や不在に気が付かない。とにかく私は、たばこを吸っている間は安易な刺激に屈することなく豊かな無為の時間を取り戻すことができるのだ。

そんな風に書き連ねれば聞こえが多少良いことを期待するが、ぼーっとするなら散歩にでも出かけて鴨川の澱みなんかを見ていた方がずっと体に良いだろう。別にこんなもの吸わない方がよい。副流煙は有害だし、においが迷惑だ。私はただただ、いつぞやの「かっこいいですね」なんてお世辞を真に受けて吸い続けている。それだけ。
